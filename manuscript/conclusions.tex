В данной работе была теоретически исследована многочастичная запутанность методами МК спектроскопии ЯМР в нанополости,
заполненной спин-несущими атомами (молекулами),
и в зигзагоборазной цепочке ядерных спинов в кристалле гамбергита.

Для нанопоры была разработана теория МК ЯМР при низких температурах.
В основе теории лежит идея о том, что молекулярная диффузия спин-несущих частиц существенно быстрее,
чем время флип-флоп процессов.
В результате задача сводится к системе эквивалентных спинов,
которая может быть проанализирована на основе общих собственных состояний полного углового момента спина и его проекции на внешнее магнитное поле.
Разработанная теория позволила исследовать динамику МК когерентностей ЯМР в системе более 200 частиц.
Поскольку удвоенный второй момент (дисперсия) распределения интенсивностей МК когерентностей ЯМР определяет нижнюю границу информации Фишера,
которая в свою очередь связана с многочастичной запутанностью,
удалось получить оценку снизу количества запутанных частиц в системе.
Температурная зависимость многочастичной запутанности была исследована для термодинамически равновесного  и дипольно упорядоченного
начальных состояний.
Теоретически было доказано,
что дипольное упорядочение состояние может быть создано двух-импульсной последовательностью Броекарта-Джинера
даже в случае низких зеемановских и высоких дипольных температур.
Несмотря на то, что начальное состояние системы не является запутанным,
при достаточно низких температурах за короткий промежуток времени МК эксперимента ЯМР почти все частицы оказываются в коллективном запутанном состоянии.

Широко представленные в литературе однородные цепочки ядерных спинов имеют узкий МК спектр ЯМР,
поэтому в таких системах удается регистрировать только парную запутанность.
Главным ограничением роста запутанных кластеров является слабое взаимодействие частицы с ее следующими ближайшими соседями.
В этом контексте особой интерес вызывают зигзагообразные цепочки в кристалле гамбергита.
При определённых ориентациях кристалла на подготовительном периоде МК эксперимента ЯМР в зигзагобразной цепочке,
в отличие от однородной, возникают МК когерентности плюс/минус 4 порядка.
Численный анализ МК динамики ЯМР зигзагообразной цепочки позволил
исследовать зависимость запутанности от температуры и длины цепочки.
Поведение температурной зависимости многочастичной запутанности в зигзагообразной цепочке качественно совпадает с поведением в нанопоре.
Результаты исследования запутанности в однородных цепочках полностью согласуются с результатами, представленными в литературе.

Величина косой информации Вигнера-Янасе,
так же как и величина квантовой информации Фишера,
определяет нижнюю границу количества запутанных частиц в системе.
Несмотря на то, что косая информация была введена задолго до квантовой информации Фишера,
и нашла широкое применения в квантовой теории информации,
её связь с наблюдаемыми в эксперименте величинами не была представлена.
В данной работе впервые предложена теория экспериментального измерения косой информации Вигнера-Янасе
в МК эксперименте ЯМР.
Полученный результат позволил провести сравнение оценок количества запутанных частиц в нанопоре и зигзагообразной цепочке,
извлеченных из величин обеих информаций.
В силу того, что величина информации Вигнера-Янасе всегда меньше квантовой информации Фишера,
полученные оценки были хуже для первой.
Тем не менее, разработанный метод имеет ряд преимуществ в сравнении с методом определения квантовой информации Фишера.
Во-первых, он позволяет определять точное значение косой информации, а не ее нежнюю границу.
Во-вторых, он является более экспериментально доступным, так как температура исследуемой системы должна быть в два раза выше.


По результатам работы можно заключить,
что МК спектроскопия ЯМР является эффективным методом исследования многочастичной запутанности,
а также может быть использована для экспериментальных исследований проблем квантовой теории информации в твердых телах.
