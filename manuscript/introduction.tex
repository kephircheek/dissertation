\chapter*{Введение}
\addcontentsline{toc}{chapter}{Введение}

% Во введении к диссертации определяется актуальность избранной темы, степень ее разработанности, цели и задачи, объект и предмет исследования, научная новизна, теоретическая и практическая значимость работы, методология диссертационного исследования, положения, выносимые на защиту, степень достоверности и апробация результатов.

% актуальность темы исследования
%   - Квантовые технологии.
%   - Квантовое превосходство.
%   - Квантовая теория информации.
%   - Запутанность в фундаментальных исследованиях.
%   - Квантовые симуляторы и квантовые компьютеры
%
% - степень ее разработанности
%   - История бинарной запутанности
%   - Критерии бинарной запутанности
%   - Эксперименты с бинарной запутанностью
%   - ?? Работы по многочастичной запутанности
%
% - цели и задачи
%   - Разработать теорию МК динамики для системы эквивалентных спинов.
%   - Исследовать многочастичную запутанность в системе эквивалентых спинов.
%   - Исследовать многочастичную запутанность в системе эквивалентных спинов с дипольно упорядоченным начальным состоянием.
%   - Исследовать многочастичную запутанность в одномерной системе.
%   - Разработать экспериментальный метод для определения информации Вигнера-Янасе в МК экcперименте ЯМР.
%   - Провести сравнение результатов оценки многочастичной запутанности на основе информации Вигнера-Янасе и информации Фишера.
%
% - научная новизна
%   - Оценка количества запутанных спинов
%   - Определение информации Вигнера-Янасе
%
% - теоретическую и практическую значимость работы;
%
% - методологию и методы исследования; ??
%
% - положения, выносимые на защиту;
%
% - степень достоверности и апробацию результатов. ??
%
%
% Nowadays, it has been recognized that most physical
% processes in nature can be formulated in terms of processing of information, and information may be central
% to understanding quantum theory [19].
%
%
% проблема классификации и количественной оценки запутанности в целом на сегодняшний день все еще далека от полного понимания.

% Such an investigation of many-spin entanglement is performed for the first time.
