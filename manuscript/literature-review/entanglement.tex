\section{Квантовые корреляции}

\subsection{Парадокс Эйнштейна — Подольского — Розена}
% Важным этапом в истории развития квантовой теории является объяснение корпускулярно-волнового дуализма.

\subsubsection{Историческая справка}

Важнейший вклад в развитие корпускулярной теории света был сделан Исааком Ньютоном.
В 1704 году им был опубликован трактат ``Оптика'',
в котором рассматриваются фундаментальные законы,
касающееся прохождения света при преломлении через призмы и линзы, дифракция, интерференция.
Несмотря на сложности связанные с описанием дифракции в рамках корпускулярной теории, которой он посветил вторую и третью часть своего трактата,
Ньютон оставался ярким сторонником корпускулярной теории.
Его работа в последствии определила основные пути развития оптики.

В 1815 году Огюстен Жан Френель дополнил принцип Христиана Гюйгенса,
описывающий механизм распространения вторичных волн,
введя представления о когерентности и интерференции элементарных волн.
Данный результат позволил легко объяснить явление дифракции,
и поставил под сомнение корпускулярную теорию света,
которая на тот момент оставалась главенствующей.
В 1818 году сторонник корпускулярной теории Симеон Дени Пуассон
в рамках волновой теории теоретически доказал существование яркого пятна,
возникающего за непрозрачным телом,
освещённым направленным пучком света,
в области его геометрической тени.
Абсурдность результата предполагалось использовать как аргумент против принципа Гюгенса-Френеля,
однако Доминик Араго поставил этот эксперимент.
Результаты эксперимента подтвердили предсказание,
а пятно Пуассона оказалось весомым аргументом в пользу новой волновой теории.

В 1873 году Джеймс Клерк Максвелл в знаменитом «Трактате об электричестве и магнетизме»
привел математическую форму для описания эффекта электромагнитной индукции,
открытого Майклом Фарадеем в 1831 году.
Из полученных Максвеллом уравнений для напряженности магнитного поля вытекало, что в пустом пространстве может распространяться электромагнитная волна, и что её скорость равна скорости света.
Эти рассуждения позволили Максвеллу сделал вывод об электромагнитной природе света.
В 1888 году в свет вышла фундаментальная работа Генриха Рудольфа Герца «Об электродинамических волнах в воздухе и их отражении»,
которая подтвердила гипотезу Максвелла.

В 1900 лорд Рэлей на основе теоремы о равнораспределении энергии по степеням свободы
получил закон распределения энергии излучения в спектре абсолютно чёрного тела в зависимости от температуры.
Закон Рэлея — Джинса правильно описывал низкочастотную часть спектра,
но при средних и высоких частотах приводил к резкому расхождению с экспериментом.
Решение ``ультрафиолетовой катастрофы'' предложил Макс Планк.
Предположив, что энергия изменяется порциями, то есть квантуется,
им был получен закон, который достоверно описывал спектральную плотность излучения абсолютно чёрным телом.

В 1902 году Филипп Ленард на основе результатов экспериментов по фотоэффекту заключил,
что, вопреки волновой теории света,
энергия вылетающего электрона всегда строго связана с частотой падающего излучения и практически не зависит от интенсивности облучения.
В 1905 году Альберт Эйнштейн объяснил теорию фотоэффекта основываясь на гипотезе Макса Планка о квантовании энергии.
В своей работе Альберт Эйнштейн постулировал,
что с электронами в веществе взаимодействуют отдельные кванты света,
обладающие свойствами частиц.
В последующих своих работах Эйнштейн подчеркивал важность применения принципа корпускулярно-волнового дуализма.
В 1926 году химик Гилберт Льюис ввел термин для кванта света --- ``фотон''.

В 1923 году Луи де Бройль развивая представления о двойственной корпускулярно-волновой природе света
выдвинул гипотезу об универсальности корпускулярно-волнового дуализма.
Он утверждал, что не только фотоны, но и электроны и любые другие частицы материи наряду с корпускулярными обладают также волновыми свойствами.
Вскоре Джордж Томсон и Клинтон Джозеф Дэвиссон с Лестером Джермером независимо обнаружили дифракцию электронов, дав тем самым убедительное подтверждение реальности волновых свойств электрона и правильности квантовой механики.

В 1927 году Вернером Гейзенбергом был сформулирован принцип устанавливающий предел точности одновременного определения пары характеризующих систему квантовых наблюдаемых. Этот результат и предположение Макса Борна о том,
что законы квантовой механики оперируют с вероятностями событий,
легли в основу Копенгагенской интерпретации квантовой механики.

В 1935 году группой авторов во главе с Альбертом Эйнштейном был предложен мысленный эксперимент,
демонстрирующий нарушающие принципа неопределенности Гейзенберга.
В последствии этот эксперимент получит название ``ЭПР-парадокс''.
В том же году Эрвин Шрёдингер поддержал Эйнштейна
и опубликовал мысленный эксперимент,
который в настоящее время известен как ``Кот Шрёдингера''.

%В последствии вероятностный исход измерения квантового состояния будет обыгран Щредингером. Он предложит мысленный эксперимент для наглядного противоречия вероятностной модели квантовой механики.
%Сегодня этот эксперимент является извейстнейшей демонстрации реального устройства измерения суперпозициии квантовых состояний.

\subsubsection{Нарушение принципа локального реализма}
Работа Эйнштейна -- Подольского -- Розена\cite{Einstein1935}
указывала на неполноту квантовой механики с помощью мысленного эксперимента,
заключающегося в измерении параметров микрообъекта косвенным образом,
без непосредственного воздействия на этот объект.

Допустим, что в определенный момент времени рождается пара фотонов $A$ и $B$,
движущихся в противоположном направлении,
с общей нулевой поляризацией.
Согласно Копенгагенской теории до измерения поляризация фотонов не определена.
Пара фотонов находится в когерентном состоянии $\ket{\Psi}$,
которое является суперпозицией двух возможных состояний:
\begin{equation}\label{eq:entangled-state}
  \ket{\Psi} = \dfrac{
    \ket{\curvearrowright_A \curvearrowleft_B}
    + \ket{\curvearrowleft_A \curvearrowright_B}
  }{\sqrt{2}}
\end{equation}
Если теперь измерить состояние одного из фотонов,
то второй фотон,
как бы он далеко не был,
мгновенно подстроится.
Состояние второго фотона будет определенным.
В своей работе\cite{Einstein1935} авторы заключают,
что из Копенгагенской теории следует,
что существует дальнодействующее взаимодействие между фотонами,
распространяющееся быстрее скорости света.

Этот мысленный эксперимент долгое время был аргументом
в пользу теории скрытых параметров.
Эйнштейн был уверен,
что никакой неопределенности нет,
и что фотоны на самом деле всегда имеют детерминированную поляризацию.
В 1964 году Джон Стюарт Белл сформулировал неравенства\cite{Bell1964},
проверяющие,
что введение дополнительных параметров не может сделать описание квантовой механики детерминированным.
Неравенство Белла показывают,
что определенные статистические корреляции,
предсказываемые квантовой механикой для измерений на двухчастичных ансамблях,
не могут быть поняты в рамках реалистической картины,
основанной на локальном реализме\cite{Einstein1935}.

В 70-е годы были проведены первые эксперименты\cite{Alain1976} Джоном Клаузером и Аленом Аспе для проверки неравенств Белла,
а 2008 году был проведен комплексный эксперимент\cite{Scheidl2010},
который окончательно подтвердил нелокальный характер квантовой теории.

В 2010 году Джон Клаузер, Ален Аспе и Антон Цайлингер стали лауреатами премии Вольфа по физике ``за фундаментальный концептуальный и экспериментальный вклад в основы квантовой физики, в частности, за серию возрастающих по сложности проверок неравенств Белла с использованием запутанных квантовых состояний''.

В действительности  ЭПР-парадокс не является парадоксом,
а скорее примером контринтуитивной природы квантовой механики.
При измерении фотона $A$ в состоянии~(\ref{eq:entangled-state}),
несмотря на то, что состояние фотона $B$ становится детерминированным,
передачи информации не происходит.
Наблюдатель фотона $B$ не будет знать его поляризацию,
не произведя измерения.
Однако результат его измерения будет детерминированным,
и может быть предсказан наблюдателем измеренного фотона $A$.

В современной науке состояния типа~(\ref{eq:entangled-state}) называются запутанными состояниями, а также состояниями Белла.


\subsection{Многочастичная запутанность}
% Phys. Rev. A 85, 022321 (2012)  B. Multiparticle Entanglement

%We give now a definition of many-particle entanglement  \cite{fisher_and_entanglement}.
%A pure state is $k$-particle entangled, if it can be written as a product \mbox{$\left| \Psi_\mathrm{k-ent} \right\rangle = \otimes^M_{l=1} \left| \Psi_l \right\rangle$}, where $\left| \Psi_l \right\rangle$ is a state of $N_l$ particles \mbox{($\sum\limits_{l=1}^M N_l = N$)}, each  $\left| \Psi_l \right\rangle$ does not factorize, and the maximal $N_l \geq k$. A generalisation for mixed states is straightforward \cite{fisher_and_entanglement}.

Многочастичная запутанность не имеет строгого определения.
В данной работе сделан акцент на количественной оценке числа запутанных частиц в системе,
поэтому удобно следовать классификации из работ~\cite{Seevinck2001, Toth2005, Bancal2009, Chen2005, Guhne2005, Guhne2006}.
Существуют альтернативные~\cite{Dur1999, Dur2000} способы классификации запутанности,
но в данной работе они рассмотрены не будут.

\begin{definition}\label{def:manyparticle-entanglement}
  Чистое состояние $N$ частиц является $k$-частино запутанным если
  $$
  \left| \Psi_{k-\mathrm{ent}} \right\rangle
  	= \otimes^\mathrm{M}_{i=1} \left| \Psi_{i} \right\rangle,
  $$
  где $\left| \Psi_{i} \right\rangle$ это несепарабельное состояние подсистемы с $N_i$ частиц
  $\left( \sum_{i=1}^N N_i = N \right)$,
  и существует такое  $ m \in \mathbb{N}$ что $N_{m} \ge k$.
  Смешанное состояние $\rho_{k-\mathrm{ent}}$ может быть представлено как
  $$
  \rho_{k-\mathrm{ent}} =
  \sum\limits_{l} p_l \ket{\Psi_{k_l-\mathrm{ent}}}\bra{\Psi_{k_l-\mathrm{ent}}},
  $$
  где $k_l \geq k$ для любого $l$.
\end{definition}


Проиллюстрируем классификацию многочастичной запутанности на примере системы из $N = 3$ частиц.
Состояние
$\ket{\Psi_\mathrm{no-ent}} = \ket{\varphi}_1 \otimes \ket{\phi}_2 \otimes \ket{\chi}_3$
является полностью сепарабельным.
Состояние
$\ket{\Psi_{2-\mathrm{ent}}} = \ket{\varphi}_{12} \otimes \ket{\chi}_3$
является двух частично запутанным,
так как $\ket{\varphi}_{12}$ не факторизуется
$\ket{\varphi}_{12} \neq \ket{\varphi}_1 \otimes \ket{\phi}_2$.
Не сепарабельное состояние $\ket{\Psi_{3-\mathrm{ent}}}$ является трех частично запутанным.




% \subsection{Приложения запутанности}
% \subsection{Использование запутанности}
% \subsection{Запутанность в фундаментальной науке}
% телепортация
% кватновая криптография
% квантовый компьютер


% В современной науке запутанность стала ключевым концептом квантовой механики.
% Также с практической точки зрения запутанные состояния
% нашли многочисленные применения в квантовой информации\cite{Gisin2002}.
%
% В экспериментах распада частиц:
% Можно еще отметить эксперимент\cite{Bernabeu2012}  Фернандо Мартинес-Видаль демонстрирующий асимметрию времени.
% Он основан на запутанности B+-мезонов и их античастиц – B--мезонов.
%
% Разбирая данные о миллиардах столкновений, накопленные почти за 10 лет, ученые обнаружили,
% что определенные типы частиц превращаются друг в друга с нарушением симметрии.
% Эти изменения чаще происходят в одну сторону, чем в другую.
% Это подтверждает, что для некоторых субатомных процессов существует направленность времени.
%
% Более того сейчас предполагается,
% что запутанность используется в ряде биологических процессов,
% одноко не доказано, является ли это эволюционным достижением или просто совпадением.


\subsection{Методы детектирования запутанных состояний}
Ввиду широкого распространения запутанности как важного ресурса,
естественно возникает вопрос о методах детектирования запутанных состояний.
Первый эффективный инструмент,
использующийся для определения запутанных состояний это неравенства Белла\cite{Bell1964}.

Существуют различные неравенства типа Белла, используемые для обнаружения запутанных состояний\cite{Collins2002, Seevinck2001, Toth2005, Nagata2002, Yu2003, Laskowski2005, Schmid2008, Bancal2009, Svetlichny1987, Gisin1998}.
Например, для некоторого набора спиновых наблюдаемых можно воспользоваться теоремой Гисина\cite{Gisin1991},
которая утверждает, что
все запутанные двухквантовые чистые состояния нарушают неравенство Клаузера-Хорна-Шимони-Холта\cite{Clauser1969}.

Были разработаны свидетели запутанности \cite{Wootters1998, Bourennane2004, Kaszlikowski2008, Krammer2009, Bancal2011},
%Например, критерий Вуттеса\cite{Wootters1998}.
%
% Phys. Rev. A 85, 022321 (2012) Intorduction
а так же критерии на основе неравенств ``сжимания'' спинов (spin-squeezing)\cite{Sorensen2001, Durkin2005, Vitagliano2011, Duan2011}.
Недавно другие подходы привели к критериям, которые могут быть оценены непосредственно по элементам матрицы плотности\cite{Guhne2010, Huber2010}.
Дальнейшие работы по обнаружению многочастичного запутывания можно найти в работах\cite{Li2010, Jungnitsch2011, Vicente2011, Huber2011} и в обзоре\cite{Guhne2009}.
В частности, квадратичные неравенства типа Белла были получены Уффинком\cite{Uffink2002} в качестве тестов на многочастичную запутанность
и используются для классификации всех состояний $N$~кубитов на $N-1$~классов запутанности от двухчастично запутанного и до полностью запутанного
 ($N$-запутанного) состояния\cite{Yu2003}

Хотя существуют и некоторые другие меры запутанных состояний\cite{Guhne2009}, проблема классификации и количественного измерения
запутанности в целом все еще далека от полного
понимания.

% Этот эффект хорошо изучен экспериментально, в качестве примера
% на слайде приведен результат из работы нашего института.
% Это температурная зависимость величины запутанности
% подсчитаной на основе согласованности в нитрозильном комплексе железа.
% Линии это теоретический расчет, точки это эксперимент.
%
% Такие результаты удается получит благодаря тому, что согласованность
% Вуттерса связана с магнитной восприимчивостью атиферомагнитного димера.


% \subsection{Экспериментальные методы детектирования запутанности}
% Статья про нитрозильные комлексы

%\subsection{Методы создания запутанных состояний}
%запутанность при передачи кванового состояния (SPIE-2019)



\subsubsection{Оценка количества запутанных частиц в системе}
\label{sec:manyparticle-entanglement-criteria}
% Phys. Rev. A 85, 022321 (2012)  B. Multiparticle Entanglement
В то время как структура множества запутанных двучастичных квантовых состояний достаточно хорошо изучена,
о классификации и количественной оценке запутанности многочастичных квантовых состояний известно меньше\cite{Plenio2007, Amico2008, Horodecki2009, Guhne2009}.
%
В данной работе будет подробно рассмотрен критерий,
использующий  неравенства Белла\cite{Bell1964} в терминах обобщенной меры информации $F$.

\begin{definition}\label{def:f}
Обобщенной мерой информации называется такая функция $F$,
которая удовлетворяет следующим свойствам:
\begin{enumerate}
  \item Значение меры объединения двух независимых систем это сумма значений меры вычисленной для каждой системы индивидуально:
  \begin{equation}\label{eq:f-additive-map}
    F(\rho_1 \otimes \rho_2 ,H_1 \otimes I_2 + I_1 \otimes H_2)
    = F_{1} (\rho_1, H_1) + F_{2} (\rho_2 , H_2),
  \end{equation}
  где $H_i$ --- это оператор действующий на подсистему $\rho_i$,
  а $I_i$ --- это единичный оператор.

  \item Известна верхняя граница величины меры для системы $N$ частиц:
  \begin{equation}\label{eq:f-supremum}
    F \leq N^2
  \end{equation}
\end{enumerate}
\end{definition}

Интерес к обобщенной мере информации $F$ вызван тем,
что для множества состояний $\mathcal{N} = \left\{\rho_{k-\mathrm{prod}}\right\}$ системы $N$ частиц,
в которых максимальный размер несепарабельной подсистемы равен $k < N$,
можно улучшить оценку (см. свойство \ref{eq:f-supremum}) ее верхней границы.
Более того ниже будет показано,
что верхняя граница обобщенной меры $F$ связана с количеством запутанных частиц в системе.

Рассмотрим некотое состояние $\rho_{k-\mathrm{prod}} \in \mathcal{N}$ ,
которое может быть факторизованно на $l$ несепарабельных подсистем $\{\rho_i\}$ с размерами $\{N_i\}$, так что:
\begin{equation}
  N > k \geq N_1 \geq N_2 \geq \dots \geq N_l, \quad \sum_{i=0}^{l} N_i = N.
\end{equation}
%
Используя свойство аддитивности~(\ref{eq:f-additive-map}),
перейдем от всей системы к рассмотрению подсистем.
%
\begin{equation}\label{eq:f-subsystems}
  F(\rho_{k-\mathrm{prod}}) =
  F(\rho_1) + F(\rho_2) + \dots + F(\rho_l)
  \leq N^2_1 + N^2_2 + \dots + N^2_l
\end{equation}
%
Максимальное значение правой части выражения~(\ref{eq:f-subsystems}) достигается,
когда cистема состоит из $m = \left[\frac N k \right]$ подсистем размера $k$
и одной подсистемой с остатком.
%
\begin{equation}\label{eq:f-worse-distribution}
  \sup_{\left\{\rho_{k-\mathrm{prod}}\right\}}
    \left(N^2_1 + N^2_2 + \dots + N^2_l\right)
  = \underbrace{
    k^2 + k^2 + \dots + k^2
    }_{m = \left[\frac N k \right] \, \mbox{раз}}
    + \underbrace{(N-km)^2}_{\mbox{остаток}}
\end{equation}
%
Докажем это утверждение для $k > 2$.
Рассмотрим состояние $\bar\rho_{k-\mathrm{prod}}$ системы  $N=k+1$ частиц.
Согласно~(\ref{eq:f-worse-distribution})
\begin{equation}
  \sup(F(\bar\rho_{k-\mathrm{prod}})) = k^2 + (N - k)^2 = k^2 + 1
\end{equation}
Предположим что это неверно,
и максимальное значения достигается, когда размер подсистемы меньше максимального размера $k$:
\begin{equation}
  k^2 + 1 < (k-1)^2 + (N - k + 1)^2 < k^2 - 2k + 1 + 2^2 < k^2 - 2k + 5 < k^2 + 1
\end{equation}
Противоречие.

Объединяя выражения~(\ref{eq:f-subsystems})~и~(\ref{eq:f-worse-distribution}),
получаем верхнюю границу величины меры $F$ для cистемы $N$ частиц с $k$-частичной запутанностью
%
\begin{equation}\label{eq:f-entangled}
  F_{N, k-\mathrm{ent}} = \left[ \frac N k \right] k^2 + \left(N - k \left[ \frac N k \right]\right)^2.
\end{equation}
%
Покажем, что правая часть~(\ref{eq:f-entangled}) не превосходит $N^2$.
Представим целую часть, как
\begin{equation}
  \left[\frac{N}{k}\right] = \frac{N}{k} -\phi,\quad 0 \leq \phi < 1,
\end{equation}
тогда
%
\begin{multline}
  k^2\left[\frac{N}{k}\right] + (N-k\left[\frac{N}{k}\right])^2
  = k^2\left(\frac{N}{k} - \phi\right) + (N - N +k\phi)^2 \\
  = kN - k^2 \phi +k^2 \phi^2 = kN - k^2 \phi(1- \phi) < kN \leq N^2
\end{multline}


Для примера ниже приведены значения $F_{N, k-\mathrm{ent}}$
для систем $N = 2, 3, 4, 5$ частиц.
\begin{itemize}
  \item[(1)]
    $F_{2, 1-\mathrm{ent}}= 2$,
    $F_{2, 2-\mathrm{ent}}= 4$;
  \item[(2)]
    $F_{3, 1-\mathrm{ent}}= 3$,
    $F_{3, 2-\mathrm{ent}}= 5$,
    $F_{3, 3-\mathrm{ent}}= 9$;
  \item[(3)]
    $F_{4, 1-\mathrm{ent}}= 4$,
    $F_{4, 2-\mathrm{ent}}= 8$,
    $F_{4, 3-\mathrm{ent}}= 10$,
    $F_{4, 4-\mathrm{ent}}= 16$;
  \item[(4)]
    $F_{5, 1-\mathrm{ent}}= 5$,
    $F_{5, 1-\mathrm{ent}}= 9$,
    $F_{5, 1-\mathrm{ent}}= 13$,
    $F_{5, 1-\mathrm{ent}}= 17$,
    $F_{5, 1-\mathrm{ent}}= 25$;
\end{itemize}


\begin{definition}
  Состояние $\rho$ является $k$-частично запутанным,
  если оно удовлетворяет неравенству
  \begin{equation}\label{eq:entanglement-criteria}
    F_{N, (k-1)-\mathrm{ent}} < F(\rho) \leq  F_{N, k-\mathrm{ent}},
  \end{equation}
\end{definition}

В настоящее время признано,
что большинство физических процессов в природе можно сформулировать в терминах обработки информации,
и концепция информации может быть центральной для понимания квантовой теории\cite{Wheeler2004, Summhammer2004, Frieden2004}.
В качестве обобщённой информационной меры в квантовой теории информации,
могут выступать
\begin{enumerate}
  \item Косая информация Вигнера-Янасе\cite{Zeqian2005},
  \item Квантовая информация Фишера\cite{Hyllus2012}.
\end{enumerate}

% В частности, было показано\cite{Garttner2018},
% что нижняя граница квантовой информации Фишера соответствует
% второму моменту спектра интенсивностей многоквантовых когерентностей.