\chapter{Многоспиновая запутанность в зигзагообразной цепочке}
% AMR-2020


% \begin{figure}
%   \includegraphics[width=0.5\textwidth]{model-zigzag-chain-schema.png}
%   \caption{Схема зигзагообзаной цепочки}
% \end{figure}
%
%     Константы взаимодействия
%         $$D_1=\dfrac{\gamma^2\hbar }{r^3}, $$
%         $$D_2 = D_1\dfrac{ 3\cos^2 \varphi -1 }{2} $$
%         $$D_3 = D_1 \dfrac{ 3\sin^2 \frac{\varphi}{2} -1}{16 \sin^3, \frac{\varphi}{2}}$$
%         где $\gamma$ - гиромагнитное отношение,
%         $\varphi$ - угол между соседними связями,
%         $r$ - расстояние между соседними спинами в цепочке.
%
%     Базовый случай это когда одна линия связи направлена вдоль поля. тогда констаты будет самой большой в доль поля.
%     Изменяя угол к полю мы можем получить как альтернированную цепочку так и однородную.
%     В приближении ближайших и следующих соседей, задача решается аналитически, но она
%     не дает полной картины. Поэтому мы решали данную систему численно.
%
%     \begin{figure}
%       \includegraphics[width=0.85\textwidth]{model-zigzag-chain-hambergite-structure.png}
%       \caption{Hанопора со спин-несущих молекулами во внешнем сильном магнитном поле $\vec B$}
%     \end{figure}
%
%     Гамбергит $Be_2BO_3(OH)$
%         \begin{itemize}
%             \item Дипольное взаимодействия между ближайшими спинами протонов в цепи в 17 раз сильнее, чем со спинами окружающих цепей (в худшем случае).
%             \item Взаимодействия с остальными окружающими спинами по меньшей мере в 30 раз слабее.
%             \item Вклад дипольной связи между спинами в одной и той же цепи доминирует над остальными взаимодействиями.
%         \end{itemize}
%
%     В одномерных цепочках возникают когерентности только $\pm 2$ порядка
%     и следовательно дисперсия распределения будет небольшой
%     и мы не увидим запутанных кластеров.
%     Однако в альтернированная цепочке гамбергита возникают когерентности $\pm 4$ порядка
%     и следовательно можно использовать эту модель для исследования многочастичной запутанности.
%     The distance to these two protons is 4.49~\r{A}
%     The distance between a given chain and surrounding proton chains is at least 2.1 times larger than the distance between neighbors in the chain.
%
