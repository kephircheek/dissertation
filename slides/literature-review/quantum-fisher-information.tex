\begin{frame}{Квантовая информация Фишера}
  \begin{block}{}
    Квантовая информация Фишера\footnote[frame]{D. Petz and C. I Ghinea, \textit{Quantum Probab. Relat. Top.} (2011)}
    показывает,
    как быстро изменяется квантовое состояние,
    определяемое матрицей плотности,
    при изменении некоторого параметра.
  \end{block}
  Квантовая информация Фишера --- это прямой аналог классической информации Фишера и основная мера в квантовой метрологии.
  Квантовая информация Фишера $F_Q$ для матрицы плотности $\rho$ по отношению к наблюдаемой $A$:
  $$
    F_Q \left(\rho, A \right) =
      2\sum_{j,k} \frac{\left(\lambda_j - \lambda_k \right)^2}
        {\lambda_j + \lambda_k}
      \left| \langle j|A|k \rangle \right|^2,
  $$
  где $\lambda_k$ и $|k \rangle$ --- собственное значение и собственный вектор матрицы плотности $\rho$.
\end{frame}
\note{
Определение: Квантовая информация Фишера [qfi](QFI) показывает, как быстро из- меняется квантовое состояние, определяемое матрицей плотности, при изменении некоторого параметра. Как правило этот параметр является временем эволюции квантового состояния.
}