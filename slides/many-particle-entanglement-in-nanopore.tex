\begin{frame}{Эволюция информации Фишера (PRA-19)}
  \begin{figure}
    \begin{subfigure}[t]{0.3\textwidth}
      \includegraphics[width=\textwidth]{m2_t_b01.pdf}
      \caption{$\beta = 0.1$}
    \end{subfigure}
    \begin{subfigure}[t]{0.3\textwidth}
      \includegraphics[width=\textwidth]{m2_t_b05.pdf}
      \caption{$\beta = 0.5$}
    \end{subfigure}
    \begin{subfigure}[t]{0.3\textwidth}
      \includegraphics[width=\textwidth]{m2_t_b3_5.pdf}
      \caption{$\beta = 3.5$}
    \end{subfigure}
    \caption{Зависимость нижней границы информации Фишера от времени.}
  \end{figure}
\end{frame}


\begin{frame}{Температурная зависимость многочастичной запутанности (PRA-19)}
\begin{columns}
    \column{0.5\textwidth}
    \begin{figure}
    \includegraphics[width=0.7\textwidth]{k_b.pdf}
    \caption{Зависимость числа запутанных частиц от обратной температуры.}
    \end{figure}


    \column{0.5\textwidth}
    В начальный момент времени система находится в термодинамическом равновесии:
    $$
    \rho_\mathrm{eq} = \dfrac{ e^{\beta I_z} }{Tr \left\{ e^{\beta I_z} \right\} },
    \quad
    \beta = \dfrac{\hbar \omega_0}{k_{B} T}
    $$
\end{columns}
\end{frame}
\note{
При понижении температуры количество запутанных частиц стремительно растет.

При достаточно низкой температуре почти все спины запутаны друг с другом,
даже когда начальное состояние не было запутанно.

}


\begin{frame}{Дипольное упорядочение (JETP-20)\footnote{I.D. Lazarev and E.B. Fel'dman , \textit{JETP} \textbf{131}, 5, (2020)}}
  \alert{При дипольно упорядоченном начальном состоянии когерентности возникают быстрее:}
  $$
  \rho_i = \frac{1}{Z_i} e^\frac{\hslash\beta_\mathrm{d} \hdz}{k}
  \approx \frac{1}{Z_i}(1 + \frac{\hslash\beta_\mathrm{d}}{k} H_\mathrm{dz}),
  $$
  то есть зеемановская температура низкая, а дипольная --- высокая.
  \begin{block}{Методы создания дипольно упорядоченного состояния}
    \begin{itemize}
      \item  Адиабатическое размагничивание\footnote[frame]{C. P. Slichter and W. C. Holton, Phys. Rev. 122, 1701 (1961)} от большого значения приложенного поля до нуля во вращающейся системе координат.

      \item Двухимпульсный эксперимент Броекаерта-Джинира\footnote[frame]{J. Jeener and P. Broekaert, Phys. Rev. 157, 232 (1967).}.
  \end{itemize}
  \end{block}
\end{frame}
\note{

Был рассмотрен случай низкой зеемановской и высокой дипольной температурами, когда начальное состояние является

%Поэтому представляется целесообразным попытаться непосредственно охладить ядерную спиновую систему, используя то обстоятельство, что она относительно изолировапа от решетки из-за слабости механизмов спин-решеточной релаксации.

%Наиболее прямой метод охлаждения ядерных моментов заключается в их адиабатическом размагничивании от большого значепия приложенного поля до пуля. При обычных условиях охлаждение, получаемое этим способом, оказывается недостаточным.

%Например, адиабатическое размагничивание ядерной спиновой системы, выполняемое, скажем, при температуре решетки $Т_0 = 1$К, начиная с поля $H0 = 25 кЭ$, приводит к конечной температуре $T_f ~ T_0 H_L / H_0$, где величина локального поля $H_L$ состав­ляет несколько эрстед. В этом случае конечная темпера­тура будет порядка $10^{-4}$ К, что по крайней мере на два порядка превосходит требуемое зпачение.

Эксперимент Броекаерта-Джинира удобнее так как это естественные

Однако подходы, разработанные для этих исследований, ограничены случаем высоких температур и не могут применяться для изучения многоспиновой запутанности.

Но нам удалось теоретически доказать что двухимпульсная последовательность Брокаерта – Джинера [17, 19] позволяет получить дипольное упорядоченное состояние даже при низкой зеемановской температуре.

Также важно отметить, что гамильтониан Hdz частично усредняется быстрой молекулярной диффу- зией в нанопоре, а усредненный гамильтониан мож- но записать как
}

% \begin{frame}{Получение дипольно упорядоченого состояния}
% В начальный момент времени система находится в термодинамическом равновесии
% %
% $$
%     \rho(0) = \rho_\mathrm{eq} = \frac{1}{Z}
%     e^{
%       \frac{\hslash \omega_{0}}{k} \alpha_\mathrm{z} I_\mathrm{z}
%       + \frac{\hslash }{k} \beta_\mathrm{d} H_\mathrm{dz}
%     }
% $$
% %
% где $Z$ --- статистическая сумма; \\
% $\alpha_\mathrm{z}$, $\beta_\mathrm{d}$ --- обратные зеемановская и дипольная температуры; \\
% $H_\mathrm{dz}$ --- секулярная часть гамильтониана ДДВ в сильном внешнем магнитном поле;
% $$
%     H_\mathrm{dz} = \frac{D}{2} (3 I^{2}_{z} - I^{2})
% $$
% Нас интересует случай,
% когда зеемановская температура низкая, а дипольная --- высокая
% $$
%     \frac{\hslash \omega_{0}}{k} \alpha_\mathrm{z}\gg 1,
%     \quad\quad
%     \frac{\hslash D}{k}\beta_\mathrm{d} \ll 1
% $$
% \end{frame}
% \note{
%     В начальный момент времени ...
%     И мы уже рассмотрели этот случай в 19 году.
%     Однако в случае дипольного упорядочения когерентности...
%     А как я говорил, дисперсия распределеяня когерентностей связана с количеством запутанных частиц.
% }


\begin{frame}{Эволюция информации Фишера (JETP-20)}
  \begin{figure}
    \begin{subfigure}[t]{0.3\textwidth}
    \includegraphics[width=\textwidth]{eval-1.png}
    \caption{$T = 6 \times 10^{-4}$ K}
    \end{subfigure}
    \hfill
    \begin{subfigure}[t]{0.3\textwidth}
      \includegraphics[width=\textwidth]{eval-2.png}
      \caption{$T = 3.2 \times 10^{-4}$ K}
    \end{subfigure}
    \hfill
    \begin{subfigure}[t]{0.3\textwidth}
      \includegraphics[width=\textwidth]{eval-3.png}
      \caption{$T = 4.8 \times 10^{-5}$ K}
    \end{subfigure}
    \caption{Зависимости нижней границы квантовой информации Фишера $F_Q = 2M_2$ от безразмерного времени $D\tau$ при $N = 101$.}
  \end{figure}
\end{frame}
\note{
    Нам удалось решить динамику для этой системы и изучить систему при разных значения температуры и размера системы.
}

\begin{frame}{Дипольное упорядочение (JETP-20)}
    \begin{figure}
    \includegraphics[width=0.7\textwidth]{entangled_spins_by_n.eps}
    \caption{
      Зависимости минимального количества числа запутанных частиц,
      усредненного по времени эволюции, от температуры.
     }
    \end{figure}
\end{frame}
\note{
мы хотели поймать скачак количества запутанных состояний, поэтому рассмотрели упорядочение

Максимальное количество запутанных спинов при одинаковой температуре увеличивается, когда увеличивается число спинов в нанопоре, потому что система в нанопоре становится плотнее.
}
