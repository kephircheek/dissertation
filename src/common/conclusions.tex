\item
Разработана теория МК ЯМР в системе эквивалентных спинов s=1/2 при произвольных температурах. При низких температурах эта теория применена для расчетов многоспиновой запутанности в
нанопоре и зигзагообразной цепочке. 
Проведенные исследования позволяют заключить, что МК-спектроскопия ЯМР является тонким и полезным методом для исследования различных проблем квантовой информатики.

\item
Исследована температурная зависимость многочастичной запутанности в нанопоре с термодинамическим равновесным зеемановским и дипольно упорядоченным начальными состояниями. 
С понижением температуры количество запутанных спинов растет.
При температуре
$T = 6.856\cdot10^{-3}$~K $(\beta=3.5)$
почти все спины (до 179 из 201) запутаны. 
Можно заключить, что в типичной системе МК ЯМР при низких температурах возникают многочастичные запутанные состояния,
даже при отсутствии запутанности в начальном состоянии.  


\item
Исследована многочастичная запутанность в квазиодномерных цепочках ядерных спинов в зависимости от параметров цепи и температуры.
В однородных цепочках детектируется только парная запутанность, что согласуется с результатами, представленными в литературе.
В зигзагообразной цепочке при низких температурах почти все спины запутанны, так же как и в нанопоре.

\item
Предложен метод экспериментального измерения точного значения косой информации Вигнера-Янасе в рамках МК спектроскопии ЯМР.
Разработанный метод позволяет не только исследовать многочастичную запутанность методами МК ЯМР,
но и открывает возможность решения широкого класса задач квантовой теории информации.
