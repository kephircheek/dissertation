% Во введении к диссертации определяется актуальность избранной темы, степень ее разработанности, цели и задачи, объект и предмет исследования, научная новизна, теоретическая и практическая значимость работы, методология диссертационного исследования, положения, выносимые на защиту, степень достоверности и апробация результатов.

\textbf{Актуальность темы исследования.}
Квантовые корреляции ответственны за преимущества квантовых приборов и устройств над их классическими аналогами.
Такие корреляции отсутствуют в классической физике.
Изучение их свойств и методов управления ими является теоретической основой квантовых технологий.

Традиционно такие корреляции связывают с понятием запутанности~\cite{Einstein1935},
но в квантовой теории информации существует и более общий класс квантовых корреляций --- квантовый дискорд~\cite{Bera2017}.
Квантовый дискорд отличен от нуля~\cite{Yurishchev2011} даже в отсутствии запутанности и при высоких температурах,
тем не менее основным ресурсом
квантовой информатики~\cite{Arute2019},
квантовой криптографии~\cite{Gisin2002},
метрологии~\cite{Toth2012}
и коммуникации~\cite{Yin2017}
является запутанность.
Изучение этого ресурса --- одна из актуальнейших проблем квантовой теории информации~\cite{Nielsen2010}.
Предпринятые в данной работе попытки количественного определения запутанности мотивированы
желанием понять и количественно оценить эти ресурсы.


С фундаментальной точки зрения большой интерес вызывают квантовые процессы,
протекающие в системах многих взаимодействующих частиц,
например, термализация~\cite{DAlessio2016}, скремблирование~\cite{Hosur2016},  локализация~\cite{Alvarez2010}.
Являясь характерной особенностью квантовой механики~\cite{Schrodinger1935},
запутанность оказывается~\cite{Kaufman2016, Neill2016, Garttner2018} ключевой особенностью этих процессов.
Дальнейшее исследование таких процессов
требует развития экспериментальных методов исследования многочастичной запутанности.
До недавнего времени исследования~\cite{Horodecki2009} были ограничены изучением запутанности
и квантового дискорда между двумя подсистемами
и направлены на определение мер этих величин.
Вместе с тем более существенны не меры квантовых корреляций,
а сам факт их наличия.
В последние годы возникли~\cite{Garttner2018} методы исследования многочастичной запутанности. 
В частности, оказалось, что в рамках многоквантовой (МК) спектроскопии ЯМР в твердом теле можно существенно продвинуться в этом направлении.

% С фундаментальной точки зрения запутанность является ключевой концепцией квантовой механики~\cite{},
% и например недавно была предложена~\cite{DAlessio2016} и экспериментально проверена~\cite{Kaufman2016, Neill2016} теория термализации изолированных квантовых систем на основе квантовой запутанности.
% Также запутанность~\cite{Ball2011} фигурирует в ряде естественных биологических процессов.
% В настоящее время запутанность является базовым инструментом для изучения экспериментов распада частиц~\cite{Bernabeu2012}.
% Также было обнаружено~\cite{Ball2011}, что запутанность фигурирует в ряде естественных биологических процессов.

% % Например, продемонстрированная~\cite{Arute2019} командой Google заявка на квантовое превосходство
% % полученное на программируемом сверхпроводящем процессоре
% % обеспечивается запутанностью.
% % Продемонстрированная~\cite{YIN2017} китайской группой передача квантовых состояний между искусственным космическим спутником и Землей реализована на основе запутанных состояний.
% % В ряде экспериментов распада частиц, как в работе\cite{Bernabeu2012} демонстрирующую асимметрию времени для субатомных процессов,
% % запутанность используется как инструмент удалённого измерения частицы.
% и например недавно была предложена~\cite{DAlessio2016} и экспериментально проверена~\cite{Kaufman2016, Neill2016} теория термализации изолированных квантовых систем на основе квантовой запутанности.
% Также запутанность~\cite{Ball2011} фигурирует в ряде естественных биологических процессов.



% \textbf{Степень ее разработанности.}
% JETP-2018
МК спектроскопия ЯМР~\cite{Baum1985} уже много лет известна как эффективный метод изучения корреляций многих взаимодействующих частиц,
так как на подготовительном периоде МК эксперимента ЯМР~\cite{Baum1985} создаются многоспиновые коррелированные кластеры.
% Главным образом МК ЯМР позволяет создавать многоспиновые коррелированные кластеры.
В работах~\cite{Krojanski2004, Zobov2006, Cho2006, Bochkin2018} были исследованы процессы роста таких коррелированных кластеров и зависимости времени декогеренции от их размера.
Также была отмечена связь запутанности с эволюцией МК когерентностей~\cite{Doronin2003, Furman2008}, %Furman2009},
а в работах~\cite{Feldman2008, Feldman2012} были введены свидетели двухчастичной запутанности.
Позднее метод МК ЯМР был применен для исследования эффекта локализации~\cite{Wei2018}.%~\cite{Alvarez2010, Alvarez2013, Alvarez2015, Wei2018}.

В недавней работе Гарттнер и др. показали~\cite{Garttner2018},
что специфический класс корреляторов,
первоначально разработанных в рамках МК спектроскопии ЯМР~\cite{Baum1985},
является полезным свидетелем многочастичной запутанности.
Спектр интенсивностей МК когерентностей ЯМР,
детектируемый по окончанию МК эксперимента ЯМР~\cite{Baum1985},
позволяет оценивать величину квантовой информации Фишера,
которая, в свою очередь, связана~\cite{Toth2014} с количеством запутанных частиц в системе.

Существуют и другие методы детектирования~\cite{Guhne2009} многочастичной запутанности.
В частности, критерий на основе энтропии Реньи~\cite{Hosur2016, Fan2017}
является строгим свидетелем многочастичной запутанности для чистых состояний.
Энтропия Реньи может быть измерена экспериментально,
но для этого требуются ресурсы,
которые экспоненциально масштабируются с размером изучаемой системы,
а также возможность одночастичной адресации.
Развиваемый в данной работе критерий многочастичной запутанности на основе МК спектра ЯМР
также является экспериментально доступным~\cite{Baum1985} свидетелем запутанности,
но менее требовательным к ресурсам,
а также применимым как к открытым,
так и к изолированным квантовым системам.


% \textbf{Цели и задачи.}
\textbf{Целью данной работы} является теоретическое исследование многочастичной запутанности в системах с большим количеством частиц $(>200)$ в рамках МК спектроскопии ЯМР,
а также развитие методов экспериментального измерения величин квантовой информации Фишера и косой информации Вигнера-Янасе.

% \textbf{Задачи:}
% \begin{enumerate}
%   \item Разработать теорию МК ЯМР для системы эквивалентных спинов при произвольной температуре.
%   \item Исследовать температурную зависимость многочастичной запутанности в нанопоре.
%   \item Исследовать температурную зависимость многочастичной запутанности в нанопоре с дипольно упорядоченным начальным состоянием.
%   \item Исследовать многочастичную запутанность в квазиодномерных цепочках ядерных спинов в зависимости от параметров цепи и температуры.
%   \item Провести сравнение оценок многочастичной запутанности, полученных на основе квантовой информации Фишера и косой информации Вигнера-Янасе.
% \end{enumerate}

\textbf{Научная новизна.}
В данной работе была разработана теория МК ЯМР для нанопоры при произвольной температуре,
что позволило впервые теоретически исследовать температурную зависимость многочастичной запутанности в системе из более чем 200 взаимодействующих частиц.
Также в данной работе был разработан метод определения величины косой информации Вигнера-Янасе в МК эксперименте ЯМР.

\textbf{На защиту выносятся следующие основные результаты и положения:}
\begin{enumerate}
  \item 
  Разработана теория МК ЯМР для системы эквивалентных спинов при произвольной температуре.
  
  \item 
  Исследована температурная зависимость многочастичной запутанности в нанопоре, 
  когда система приготовлена в термодинамическом равновесном зеемановском и дипольно упорядоченном состояниях.
  
  \item
  Исследована многочастичная запутанность в квазиодномерных цепочках ядерных спинов в зависимости от параметров цепи и температуры.
  
  \item 
  Предложен метод экспериментального измерения точного значения косой информации Вигнера-Янасе в рамках МК спектроскопии ЯМР.
  
  \item 
  Проведено сравнение оценок многочастичной запутанности, полученных на основе квантовой информации Фишера и косой информации Вигнера-Янасе.
\end{enumerate}
% \begin{enumerate}
%   \item Разработана теория МК ЯМР для системы эквивалентных спинов при произвольной температуре.
%   \item Исследована температурная зависимость многочастичной запутанности в нанопоре с термодинамическим и дипольно упорядоченным начальными состояниями.
%   \item Исследована многочастичная запутанность в квазиодномерных цепочках ядерных спинов в зависимости от параметров цепи и температуры.
%   \item Предложен метод экспериментального измерения точного значения косой информации Вигнера-Янасе в рамках МК спектроскопии ЯМР.
%   \item Проведено сравнение оценок многочастичной запутанности, полученных на основе квантовой информации Фишера и косой информации Вигнера-Янасе.
% \end{enumerate}

\textbf{Практическая ценность.}
Так как косая информация Вигнера-Янасе нашла много применений в квантовой теории информации, %~\cite{Wigner1963, Luo2017},
предлагаемый в данной работе метод определения ее величины
не только позволяет исследовать многочастичную запутанность методами МК ЯМР,
но и открывает возможность решения широкого класса задач в этой области.

\textbf{Публикации и апробация работы.}
Все результаты, представленные в диссертации,
опубликованы в высокорейтинговых зарубежных и российских научных журналах (Physical Review A, 2019;  Journal of Magnetic Resonance, 2020; Журнал экспериментальной и теоретической физики, 2020; Applied Magnetic Resonance, 2020;  Physics Letters A, 2021) и представлены на пяти международных и одной всероссийской конференциях.



% Многоквантовая (МК) спектроскопия ЯМР была введена для исследования распределения ядерных спинов в различных материалах (жидкие кристаллы, простые органические системы, аморфный гидрогенизированный кремний и т.д.).
% Он также оказался полезным для исследования скорости декогеренции в сильно коррелированных спиновых кластерах \cite{decoherence_register,decoherence_ca_f2}.
% Также была продемонстрирована зависимость скорости декогеренции от числа коррелированных спинов \cite{decoherence_register,lab:decoherence_2018}.
% По сути, динамика MQ ЯМР является подходящим методом для количественной оценки развития MQ когерентности, начиная с $z$-поляризации и заканчивая коллективным состоянием всех спинов.
% Метод позволяет описать распространение корреляций \cite{mq_nmr_experiment,spin_distribution_in_liquid_system,decoherence_under_dq,nmr_dyn} и предлагает сигнатуру эффектов локализации \cite{loc_deloc_nmr_dyn,loc_in_chain}. Скорость распространения может быть описана через вневременные упорядоченные корреляции (OTOCs), которые связаны с распределением когерентностей MQ ЯМР.
%

%
%
% В современной науке запутанность стала ключевым концептом квантовой механики.





%
%
% Nowadays, it has been recognized that most physical
% processes in nature can be formulated in terms of processing of information, and information may be central
% to understanding quantum theory [19].
%
%
% проблема классификации и количественной оценки запутанности в целом на сегодняшний день все еще далека от полного понимания.

% Such an investigation of many-spin entanglement is performed for the first time.
