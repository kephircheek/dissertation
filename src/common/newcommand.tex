\newcommand{\bra}[1]{\left\langle #1 \right|}
\newcommand{\ket}[1]{\left| #1 \right\rangle}
\newcommand{\p}[1]{\left( #1 \right)}
\newcommand{\abs}[1]{\left| #1 \right|}
\newcommand{\tr}[1]{\mathrm{Tr} \left\{ #1 \right\}}
\newcommand{\commutator}[2]{\left[#1, #2\right]}

\newcommand{\soperator}{I}
\newcommand{\sx}{\soperator_\mathrm{x}}
\newcommand{\sy}{\soperator_\mathrm{y}}
\newcommand{\sz}{\soperator_\mathrm{z}}
\newcommand{\splus}{\soperator^+} % raising
\newcommand{\sminus}{\soperator^-} % lowering
\newcommand{\splusminus}{\soperator^\pm}
\newcommand{\hdz}{H_\mathrm{dz}}
\newcommand{\hmq}{H_\mathrm{MQ}}
\newcommand{\qfi}{F_\mathrm{Q}}
\newcommand{\wyi}{I_\mathrm{WY}}

\newcommand{\rhoEq}{\rho_\mathrm{eq}}
\newcommand{\rhoEqDefinition}{
  \dfrac{e^{\beta \sz}}{Z},
  \quad
  \beta = \frac{\hbar\omega_{0}}{kT}
 }
\newcommand{\rhoEqExplanatoryNote}{
  где $Z = \tr{e^{\beta \sz}}$
  --- статистическая сумма,
  $\hbar$ и $k$ --- постоянная Планка и постоянная Больцмана соответсвенно,
  $\omega_0$ --- ларморовская частота,
  $T$ --- температура,
  и $I_z$ --- оператор проекции полного углового момента на ось $z$,
  которая направлена вдоль сильного внешнего магнитного поля.
}

\newcommand{\rhoEqHdz}{\bar\rho_\mathrm{eq}}
\newcommand{\rhoEqHdzDefinition}{
  \dfrac{1}{\bar Z}
    e^{
      \frac{\hslash \omega_{0}}{k} \alpha_\mathrm{z} \sz
      + \frac{\hslash }{k} \beta_\mathrm{d} \hdz
    }
}
\newcommand{\rhoEqHdzExplanatoryNote}{
  где
  $\bar Z = \tr{ e^{\frac{\hslash \omega_{0}}{k} \alpha_\mathrm{z} \sz + \frac{\hslash }{k} \beta_\mathrm{d} \hdz} }$ --- статистическая сумма,
  $\hslash$ и $k$ --- константы Планка и Больцмана,
  $\omega_{0}$ --- частота Лармора,
  $\sz$ ---  оператор проекции полного углового спинового момента  на ось~$z$,
  который направлен вдоль сильного внешнего магнитного поля,
  $\hdz$ --- секулярная часть гамильтониана ДДВ в сильном внешнем магнитном поле
  и $\alpha_\mathrm{z}$, $\beta_\mathrm{d}$ --- обратные зеемановская и дипольная температуры.
}

\newcommand{\rhoDo}{\rho_\mathrm{do}}
\newcommand{\rhoDoDefinition}{
  \frac{1}{Z_\mathrm{do}} e^\frac{\hslash\beta_\mathrm{d} \hdz}{k}
}
\newcommand{\rhoDoDefinitionHT}{
  \frac{1}{Z_\mathrm{do}}\p{1 + \frac{\hslash\beta_\mathrm{d}}{k} \hdz}
}
\newcommand{\rhoDoExplanatoryNote}{
  где статистическая сумма
  $Z_\mathrm{do} = \tr{ e^\frac{\hslash\beta_\mathrm{d} \hdz}{k} } \approx 2^{N}$.
}

\newcommand{\rhoHT}{\rho_\mathrm{HT}}
\newcommand{\rhoHTDefinition}{

}

\newcommand{\rhoLT}{\rho_\mathrm{LT}}
\newcommand{\rhoLTDefinition}{

}

\newcommand{\liouvilleEquation}[2][\rho]{
  i \frac{d #1}{d \tau} = \commutator{#2}{#1(\tau)}
}
% $\liouvilleEquation[\rho]{\hmq}$
% $\liouvilleEquation{\hmq}$


\newcommand{\hmqEquivalentSpins}{H_\mathrm{MQ,es}}%^\mathrm{es}}
\newcommand{\hmqEquivalentSpinsDefinition}{
   \hmqEquivalentSpins = - \frac{D_\mathrm{es}}{4} \p{
    \p{\splus}^2 + \p{\sminus}^2
  },
  \quad
  \splusminus = \sum\limits_{j=1}^{N} I^{\pm}_j
}
\newcommand{\hmqEquivalentSpinsExplanatoryNote}{
  где $\splus_j$ и $\sminus_j$ --- повышающий и понижающий операторы спина $j$,
  $N$ --- число спинов в нанопоре,
  $D_\mathrm{es}$ --- константа ДДВ,
  усредненная по быстрой молекулярной диффузии спин-несущих частиц в нанопоре.
}
\newcommand{\hmqEquivaletnSpinsBlockDegeneration}{n(S, N)}
\newcommand{\hmqEquivaletnSpinsBlockDegenerationDefinition}{
  n(S, N) = \dfrac{
    N! (2S+1)
  }{
    \p{\frac N 2 + S + 1}!\p{\frac N 2 - S}!
  }%,
  %\quad
  %\frac N 2 - \left[\frac N 2\right] \leq S \leq \frac N 2
}

\newcommand{\hdzEquivaletnSpins}{H_\mathrm{dz, es}}
\newcommand{\hdzEquivaletnSpinsDefinition}{
  \frac{D_\mathrm{es}}{2} (3 \sz^2 - \soperator^2)
}
\newcommand{\hdzEquivaletnSpinsExplanatoryNote}{
  где $\soperator^2$ --- квадрат полного спинового углового момента.
}

\newcommand{\hmqZChainNextNearest}{H_\mathrm{MQ, zc}}
\newcommand{\hmqZChainNextNearestDefinition}{
  \sum_{i=1}^{N-1} D_{i, i+1}(I_i^{+}I_{i+1}^{+}+ I_i^{-}I_{i+1}^{-} )
  + \sum_{i=1}^{N-2} D_{i, i+2} (I_i^{+}I_{i+2}^{+}+ I_i^{-}I_{i+2}^{-} )
}
\newcommand{\hmqZChainNextNearestExplanatoryNote}{
  где $I_i^+$, $I_i^-$ ---  повышающий и понижающий операторы спинового углового момента ядра с номером $i$,
  $N$ ---  количество ядерных спинов в цепочке.
}

\newcommand{\totalSpinAnglularMomentumValuesDefinition}{
  S = \frac N 2,
    \frac N 2 - 1,
    \frac N 2 - 2,
    \dots,
    \frac N 2 - \left[\frac N 2\right]
}
\newcommand{\totalSpinAnglularMomentumValuesExplanatoryNote}{
  где $[x]$ --- целая часть $x$.
}
