\textbf{Объектом исследования} являются системы взаимодействующих ядерных спинов $\frac{1}{2}$ при низких температурах. 
В качестве таких систем рассматриваются тонкая пленка, содержащая нанопоры, заполнения спин-несущими частицами\cite{Baugh2001}, 
зигзагообразные цепочки протонов в кристалле гамбергита\cite{Bochkin2020jmr} 
и цепочки ядер фтора в кристалле фтористого апатита кальция\cite{Bochkin2019jmr}.   
\textbf{Предметом исследования} является запутанность возникающая в таких системах в МК эксперименте ЯМР, 
а также теория методов измерения квантовых информационных величин. 