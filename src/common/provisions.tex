\begin{enumerate}
  \item
  Разработанная теория МК ЯМР позволяет исследовать многочастичную запутанность в системе ядерных спинов при произвольной температуре.

  \item
  С понижением температуры количество запутанных спинов растет и в нанопоре, и в зигзагобразной цепочке. 
  % В нанопоре, заполненой спин несущими частицами, при температуре 
  % $T = 6.856\cdot10^{-3}$~K $(\beta=3.5)$ 
  % почти все спины (до 179 из 201) запутаны.
  %Исследована температурная зависимость многочастичной запутанности в нанопоре,
  %когда система приготовлена в термодинамическом равновесном зеемановском и дипольном упорядоченном состояниях.

  \item
  Оценка количества запутанных спинов в однородных цепочках согласуется с результатами, представленными в литературе.
  % Исследована многочастичная запутанность в квазиодномерных цепочках ядерных спинов в зависимости от параметров цепи и температуры.

  \item
  Если спиновая система исследуется в МК эксперименте ЯМР с начальным равновесным термодинамическим состоянием при температуре $T$, 
  то ее косая информация Вигнера-Янасе равна удвоенному второму моменту распределения интенсивностей МК когерентностей ЯМР системы, приготовленной при вдвое большей температуре $2T$ в тот же момент времени эволюции;
  % Предложен метод экспериментального измерения точного значения косой информации Вигнера-Янасе в рамках МК спектроскопии ЯМР.

  \item
  Результаты оценки количества запутанных спинов, полученные на основе квантовой информации Фишера и косой информации Вигнера-Янасе, согласуются;
  % Проведено сравнение оценок многочастичной запутанности, 
  % полученных на основе квантовой информации Фишера и косой информации Вигнера-Янасе.
\end{enumerate}
