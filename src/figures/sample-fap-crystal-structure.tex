(a) Структура фтористого апатита, $c$-ось направлена перпендикулярно плоскости рисунка.
Выделена одна ячейка.
Каждая цепь ядер $^{19}$F, идущая вдоль оси $c$, окружена шестью идентичными цепями на расстоянии $a=9.37$~\r{A}.
(b) Часть кристаллической структуры FAp, показывающая окружение атомов фтора. Атомы кислорода были удалены для упрощения.
Атомы фтора равномерно разнесены ($r_{FF}=344$~пкм) и расположены в колонны вдоль $c$-оси кристалла.
Каждый атом фтора окружен тремя равноудаленными атомами фосфора при $r_{FP}=3,67$~\r{A},
которые расположены в вершинах равносторонних треугольников на плоскости,
перпендикулярной к $c$-оси, а также тремя ионами CaII на расстоянии 2.34~\r{A}.
