\section{Сравнение информаций Фишера и Вигнера-Янасе}
\label{sec:qfi-wyi-comparison}
%
Квантовая информация Фишера~(см. раздел~\ref{sec:quantum-fisher-information})
и косая информация Вигнера-Янасе~(см. раздел~\ref{sec:skew-wigner-yanase-information})
удовлетворяют свойствам обобщенной информации~(см. определение~\ref{def:f}).
Следовательно, обе информации могут быть использованы для исследования многочастичной запутанности.
Для того чтобы прояснить связь между этим фундаментальными мерами,
в этом разделе будет проведено сравнение величин этих информаций для произвольной матрицы плотности $\rho$
и оператора проекции полного углового спинового момента на ось~$z$
в качестве наблюдаемой.

Матрица плотности $\rho$ в диагонализирующем ее базисе,
имеет вид
%
\begin{equation}
  \rho = \sum\limits_i \lambda_i \ket{i}\bra{i},
\end{equation}
%
где $\lambda_i$ --- собственное значение,
а $\ket{i}$ соответствующий собственный вектор.
%
В этом базисе косая информация Вигнера-Янасе может быть записана как:
%
\begin{multline}\label{eq:wyi-via-lambda}
  I_\mathrm{WY}(\rho, I_z)
  = -2 \tr{[\rho, I_z]^2}
  = -2 \sum\limits_{i,k} [\sqrt{\rho}, I_z]_{ik} [\sqrt{\rho}, I_z]_{ki}
  \\
  = -2 \sum\limits_{i,k}
    \p{\sqrt{\lambda_i}(I_z)_{ik} - (I_z)_{ik} \sqrt{\lambda_k}}
    \p{\sqrt{\lambda_k}(I_z)_{ki} - (I_z)_{ki} \sqrt{\lambda_i}}
  \\
  = -2 \sum\limits_{i,k} \p{
    \sqrt{\lambda_i}\sqrt{\lambda_k}
    - \lambda_k
    - \lambda_i
    + \sqrt{\lambda_k}\sqrt{\lambda_i}
    }
    (I_z)^2_{ki}
  \\
  = 2 \sum\limits_{i,k} \p{\sqrt{\lambda_i} -\sqrt{\lambda_k}}^2
    \left| \bra{i} I_z \ket{k} \right|^2.
\end{multline}
%
В свою очередь, квантовая информации Фишера может быть выражена как
%
\begin{multline}\label{eq:qfi-via-wyi}
  I_\mathrm{F}(\rho, I_z)
  = 2 \sum\limits_{i,k}
    \dfrac{\p{\lambda_i - \lambda_k}^2}{\lambda_i + \lambda_k}
    \left| \bra{i} I_z \ket{k} \right|^2
  \\
  = 2 \sum\limits_{i,k}
    \dfrac{
      \p{\sqrt{\lambda_i} +\sqrt{\lambda_k}}^2
    }{
      \lambda_i + \lambda_k
    }
    \p{\sqrt{\lambda_i} -\sqrt{\lambda_k}}^2
    \left| \bra{i} I_z \ket{k} \right|^2
  \\
  = 2 \sum\limits_{i,k}
    \p{1 + \dfrac{2\sqrt{\lambda_i}\sqrt{\lambda_k}}{\lambda_i + \lambda_k}}
    \p{\sqrt{\lambda_i} -\sqrt{\lambda_k}}^2
    \left| \bra{i} I_z \ket{k} \right|^2
  \\
  = 2 \sum\limits_{i,k} \p{1 + \alpha_{ik}}
  \p{\sqrt{\lambda_i} -\sqrt{\lambda_k}}^2
  \left| \bra{i} I_z \ket{k} \right|^2,
\end{multline}
%
где $\alpha_{ik} = \dfrac{2\sqrt{\lambda_i}\sqrt{\lambda_k}}{\lambda_i + \lambda_k}$.
Так как $ 0 \leq \alpha_{ik} \leq 1$
из выражений~(\ref{eq:qfi-via-wyi})~и~~(\ref{eq:wyi-via-lambda}) следует, что
%
\begin{equation} \label{eq:qfi-wyi-inequality}
    I_{WY}\left(\rho, I_z\right)
    \leq I_F\left(\rho, I_z\right)
    \leq 2I_{WY}\left(\rho, I_z\right).
\end{equation}
