\section{Информация Фишера}
% история
% oпределение
% применение
% вывод
% свойства
% формула
% связь со вторым моментом

% PETZ2011 https://doi.org/10.1142/9789814338745_0015
Оценка параметров вероятностных распределений является одной из самых основных задач в теории информации,
и она была обобщена на квантовый режим\cite{Helstrom1969, Holevo2011}, поскольку описание квантового измерения является по сути вероятностным.

\subsection{Классическая информация Фишера}
Классическая информация Фишера является мерой того,
как быстро изменяется распределение информации при изменении некоторого параметра.

\begin{definition}\label{def:fisher-information}
 Информация Фишера --- это математическое ожидание квадрата относительной скорости изменения условной плотности вероятности,
 при изменении некоторого параметра.
\end{definition}
%
\begin{equation}\label{eq:fisher-information}
  F_c({\theta})=
      \sum_x{{P}(x,\theta)}
          \left[\frac{1}{P(x,\theta)}
     \frac{d}{d\theta}
  {P(x,\theta)}\right]^2.
\end{equation}
%
Важно отметить,
что квадрат играет ключевую роль в определении информации Фишера
и отказаться от него невозможно:
%
\begin{equation}
    \label{eq:2}
        \sum_x{{P}(x,\theta)}
            \frac{1}{P(x,\theta)}
                \frac{dP(x,\theta)}{d\theta} =
                    \sum_x\frac{{dP}(x,\theta)}{d\theta} =
                \frac{d}{d\theta}\sum_x{{P}(x,\theta)} =
            \frac{d}{d\theta}1 =
            0.
\end{equation}

Выражение~(\ref{eq:fisher-information}) для классической информации фишера, можно переписать через математическое ожидание:
\begin{multline}\label{eq:fi-via-mean-conversion}
    F_c(\theta)
    = \sum_x P(x, \theta)\p{\frac{1}{P(x,\theta)}\frac{d}{d \theta} P(x, \theta)}^2 \\
    = \sum_x P(x,\theta)\p{\frac{d}{d \theta} \ln P(x, \theta)}^2
    = E\left[\p{\frac{d}{d \theta}\ln f(X,\theta)}^2 \right],
\end{multline}
%
где $f(X,\theta) = \prod_x P(x, \theta)$.
В случае, если $f(X,\theta)$ дважды дифференцируема по $\theta$,
выражение~(\ref{eq:fi-via-mean-conversion}), можно упростить и записать в виде
%
\begin{equation}\label{eq:fi-via-mean}
  F_c(\theta) = -E \left[\frac{\partial^2}{\partial \theta^2} \ln f(X,\theta)\right].
\end{equation}
%
Для доказательства выражения~(\ref{eq:fi-via-mean})
продифференцируем выражение под оператором математического ожидания:
%
\begin{multline}\label{eq:fi-via-mean-expand}
  \frac{\partial^2}{\partial \theta^2}\ln f(X,\theta)
  = \frac{\partial}{\partial \theta}\p{\frac{1}{f(X,\theta)}
    \frac{\partial f(X,\theta)}{\partial \theta} } \\
  = \frac{1}{f(X,\theta)}\frac{\partial^2 f(X,\theta)}{\partial \theta^2}
  - \p{\frac{1}{f(X,\theta)} \frac{\partial f(X,\theta)}{\partial \theta}}^2.
\end{multline}
%
Покажем,
что математическое ожидание первого слагаемого выражения~(\ref{eq:fi-via-mean-expand}) тождественно равно нулю:
%
\begin{multline}\label{eq:fi-via-mean-proof}
  E\left[\frac{1}{f(X,\theta)}
    \frac{\partial^2 f(X,\theta)}{\partial \theta^2}\right]
  = \int f (X,\theta) \frac{1}{f(X,\theta)}
    \frac{\partial^2 f(X, \theta)}{\partial \theta^2}dx \\
  = \int \frac{\partial^2 f(X,\theta)}{\partial \theta^2}dx
  = \frac{\partial^2}{\partial \theta^2} \int f(X,\theta)dx
  = \frac{\partial^2}{\partial \theta^2}1
  \equiv 0.
\end{multline}
%
Отсюда следует~(\ref{eq:fi-via-mean}).
Приведенное доказательство справедливо при условии регулярности $f(X,\theta)$.


%  (Cramer-Rao bound)
Информация Фишера играет важную роль в задачах оценки параметров.
Оказалось, что обратная информация Фишера является нижней границей дисперсии при оценке параметров, называемой границей Крамера-Рао.
Докажем это утверждение.
%
Исходим из тождества:
%
\begin{equation}\label{eq:21}
  E[\hat{\theta}(X) - \theta]
  = \int(\hat{\theta}(x) - \theta)f(x,\theta)d\theta
  = \int\theta(x)f(x,\theta)dx - \int f(x,\theta)dx \equiv 0.
\end{equation}
%
Формально это выражение может быть записано следующим образом
%
\begin{equation}\label{eq:22}
  0 = \frac{\partial}{\partial \theta}
    \int(\hat{\theta}(X) - \theta)f(x,\theta)dx
  = \int(\hat{\theta}(X) - \theta)
    \frac{\partial f(X, \theta)}{\partial \theta}dx - \int f(x,\theta)dx.
\end{equation}
%
Поскольку $\frac{\partial f}{\partial \theta} = f(x,\theta)
    \frac{\partial \ln f(x,\theta)}{\partial \theta}$,
~(\ref{eq:22}) можно переписать в виде
%
\begin{equation}
    \label{eq:23}
        \int(\hat{\theta}(X) - \theta)
            f(x,\theta) \frac{\partial \ln f(x,\theta)}
        {\partial \theta}dx = 1.
\end{equation}
%
Выражение~(\ref{eq:22}) перепишем в форме:
%
\begin{equation}
    \label{eq:24}
        \int \left[\hat{\theta}(X-\theta)\Large \sqrt{\mathstrut f(x,\theta)}\right]
            \left[\Large\sqrt{\mathstrut f(x,\theta)}
                \frac{\partial \ln f(x,\theta)}{\partial \theta}\right]dx = 1.
\end{equation}
%
Применяя к~(\ref{eq:24}) неравенство Коши-Шварца, получаем:
%
\begin{multline}
    \label{eq:25}
        1 = \p{\int \left[(\hat{\theta}(X)-\theta)
            \Large \sqrt{\mathstrut f(x,\theta)}\right]
                \left[\Large \sqrt{\mathstrut f(x,\theta)}
                    \frac{\partial \ln f(x,\theta)}{\partial \theta} \right]
                }^2 \\
                \leq \int
            \underbrace{[(\hat{\theta}(X) - \theta)^2 f(x,\theta)]}_{Var(\hat{\theta}(X))}dx
        \underbrace{\int\left[f(x,\theta)\p{\frac{\partial \ln f(x,\theta)}
    {\partial \theta}}^2 \right]dx}_{F_c(\theta)}.
\end{multline}
%
Отсюда получаем выражение для границы Крамера-Рао:
%
\begin{equation}
    \label{eq:26}
        Var(\hat{\theta}(X)) \geq 1/{F_c (\theta)}.
\end{equation}




\subsection{Квантовая информация Фишера}
\label{sec:quantum-fisher-information}
\begin{definition}\label{def:quantum-fisher-information}
  Квантовая информация Фишера~\cite{liu2014} показывает,
  как быстро изменяется квантовое состояние,
  определяемое матрицей плотности, при изменении некоторого параметра.
\end{definition}
Переход к квантовой информации Фишера соответствует обычной процедуре перехода от классических величин к квантовым.
Квантовая информация Фишера определяется по формуле:
%
\begin{equation}\label{eq:qfi-via-l}
  F_Q (\rho_\theta) = Tr[\rho_\theta, L^2_\theta],
\end{equation}
%
где $\rho_\theta$ --- это матрица плотности системы, зависящая от параметра $\theta$, а симметричный оператор логарифмической производной $L_\theta$ определяется из уравнения:
%
\begin{equation}\label{eq:qfi-log-derivative}
  \frac{\partial\rho_\theta}{\partial\theta}
  = \frac{1}{2} \p{L_\theta \rho_\theta + \rho_\theta L_\theta}.
\end{equation}
%
В классическом случае, когда $\left[L_\theta, \rho_\theta \right]$ = 0, из ~(\ref{eq:qfi-log-derivative}) находим,
что ${L_\theta\sim\frac{\partial\ln\rho_\theta}{\partial\theta}}$,
и формула~(\ref{eq:qfi-via-l}) дает результат, совпадающий с~(\ref{eq:fisher-information}).
%
Зависимость матрицы плотности системы от параметра $\theta$ может быть введена в систему разными способами.
Мы рассмотрим случай, когда эта зависимость определяется унитарной эволюцией
%
\begin{equation}%\label{eq:5}
  \rho_\theta = e^{i H \theta} \rho e^{-i H \theta},
\end{equation}
%
где $\rho$ --- произвольная матрица плотности, $H$ --- эрмитов гамильтониан системы.
Такой способ введения параметра отвечает условиям многоквантовой спектроскопии ЯМР.
В этом случае:
%
\begin{equation}\label{eq:derivative-rho-by-theta}
  \frac{\partial\rho_\theta}{\partial\theta}
  = iH\rho_\theta - i\rho_\theta H
  = i \left[H,\rho_\theta \right].
\end{equation}
%
Дальнейшие вычисления проведем в базисе, диагонализирующем матрицу плотности
$\rho =\sum\limits_{k} \lambda_k \ket{k}\bra{k}$.
Используя уравнения ~(\ref{eq:qfi-log-derivative})~и~(\ref{eq:derivative-rho-by-theta}), получаем
%
\begin{multline}%\label{eq:7}
  i \left[H,\rho_\theta \right]_{jk}
  = \frac{1}{2} \sum_e \p{L_\theta }_{je}
    \p{\rho_\theta }_{ek} +
    \frac{1}{2} \sum_e \p{\rho_\theta }_{je}
    p{L_\theta }_{ek} \\
  = \frac{1}{2}\p{L_\theta }_{jk} \lambda_k
  + \frac{1}{2}\lambda_j \p{L_\theta }_{jk}
  = \frac{1}{2} \p{\lambda_k + \lambda_j }
    \p{L_\theta }_{ik}.
\end{multline}
%
Таким образом,
 %
\begin{equation}% \label{eq:8}
  \p{L_\theta }_{jk}
  = \frac{2i\left[H,\rho_\theta \right]_{jk}}{\lambda_k + \lambda_j}.
\end{equation}
%
Умножим теперь~(\ref{eq:derivative-rho-by-theta}) на $L_\theta$ и возьмем след:
%
\begin{equation}%\label{eq:9}
  \tr{
    L_\theta \frac{\partial \rho_\theta}{\partial \theta}
  }
  = \frac{1}{2} \tr{L^2_\theta \rho_\theta}
  + \frac{1}{2} \tr{L_\theta \rho_\theta L_\theta}
  = \tr{L^2_\theta \rho_\theta}
  = F_Q \p{\rho_\theta, H}.
\end{equation}
%
Далее
%
\begin{equation}%\label{eq:10}
  \left[H, \rho_\theta \right]_{jk}
  = \sum_e H_{je} \p{\rho_\theta }_{ek}
  - \sum_e \p{\rho_\theta }_{je}H_{ek}
  = H_{jk}\lambda_k - H_{jk}\lambda_j
  = \p{\lambda_k - \lambda_j } H_{jk}.
\end{equation}
%
Теперь
%
\begin{equation}%\label{eq:11}
  F_Q \p{\rho_\theta, H }
  = \sum_{j,k}\frac{2i\left[H,\rho_\theta \right]_{jk}}{\lambda_k + \lambda_j}
    i \left[H, \rho_\theta \right]_{kj}
  = 2\sum_{j,k} \frac{\p{\lambda_j - \lambda_k }^2}{\lambda_j + \lambda_k}
    \abs{\bra{j} H \ket{k}}^2.
\end{equation}
%
В итоге получаем основную формулу для квантовой информации Фишера
%
\begin{equation}\label{eq:quantum-fisher-information}
        F_Q \p{\rho_\theta, H } =
            2\sum_{j,k} \frac{\p{\lambda_j - \lambda_k }^2}
                {\lambda_j + \lambda_k}
            \left| \langle j|H|k \rangle \right|^2.
\end{equation}

Покажем, что квантовая информация Фишера удовлетворяет свойствам~(\ref{def:f}),
и, следовательно, может быть использована для оценки количества запутанных частиц.
Квантовая информации Фишера в сепарабельной системе с матрицей плотности $\rho = \rho_1 \otimes \rho_2$ удовлетворяет свойству аддитивности:
%
\begin{equation}\label{eq:qfi-via-l1}
  F_Q(\rho_1 \otimes \rho_2 ,H_1 \otimes I_2 + I_1 \otimes H_2) =
      F_{Q1} (\rho_1, H_1) + F_{Q2} (\rho_2 , H_2).
\end{equation}
%
Также величина квантовой информации Фишера в системе из $N$ частиц
ограничена сверху\:
%
\begin{equation}\label{eq:qfi-via-l3}
  F_Q \leq N^2.
\end{equation}
%
Докажем утверждение~(\ref{eq:qfi-via-l3}).
Исходим из формулы
%
\begin{equation}\label{eq:qfi-via-l4}
  F_Q = 2\sum_{\substack{j,k \\ \lambda_j + \lambda_k \neq 0}}
  \frac{(\lambda_j - \lambda_k)^2}{\lambda_j + \lambda_k}
  \left| \langle j|H|k \rangle \right|^2.
\end{equation}
%
Так как
$\p{\dfrac{\lambda_j - \lambda_k}{\lambda_j +\lambda_k}}^2 \leq 1$,
\begin{equation}
  F_Q \leq 2\sum_{j,k} \p{\lambda_j + \lambda_k} \abs{\bra{i} H \ket{j}}^2
  \leq 4\sum_{j,k} \lambda_j \abs{\bra{i} H \ket{j}}^2
  = 4 \tr{\rho H^2}
  \leq N^2,
\end{equation}
где учтено, что $H = \sum_i H_i$,
$H_i$ --- локальная наблюдаемая и $H_i^2 = \frac 1 4$.

%
% Убедимся, что
% \begin{equation}
%   \p{\frac{(\lambda_j - \lambda_k)^2}{\lambda_j +\lambda_k}} \leq 1.
% \end{equation}
% Неравенство
% \begin{equation}
%   \lambda_j + \lambda_k \geq \lambda^2_j +\lambda^2_k
% \end{equation}
% верно, т.к. $0\leq \lambda_j \leq 1$, $0\leq \lambda_k \leq 1$.
% %
% Далее
% \begin{equation}
% \lambda_j + \lambda_k \geq\lambda^2_j +\lambda^2_k \geq \lambda^2_j +\lambda^2_k - 2\lambda_j \lambda_k = (\lambda_j - \lambda_k)^2 \Rightarrow 1 \geq \frac{(\lambda_j - \lambda_k)^2}{\lambda_j + \lambda_k}.
% \end{equation}
% %
% Гамильтониан Н в~(\ref{eq:qfi-via-l4}) безразмерный.
% Предполагаем, обезразмеривание проиведено на максимальный модуль недиагонального элемента гамильтониана Н (все элементы поделены на $\left| \langle j|H|k \rangle \right| + \left| \langle k|H|j \rangle \right| = 2\left| \langle j|H|k \rangle \right|$,)
% %
% тогда
% %
% \begin{equation}
%     \label{eq:qfi-via-l5}
%         \left| \langle j|H|k \rangle \right| \leq 1/2
% \end{equation}
% %
% В итоге получаем оценку сверху для квантовой информации Фишера:
% %
% \begin{equation}
%     \label{eq:qfi-via-l6}
%         F_Q = 2\sum_{\substack{j,k \\ \lambda_j + \lambda_k \neq 0}}
%             \frac{(\lambda_j - \lambda_k)^2}{\lambda_j + \lambda_k}
%                 \left| \langle j|H|k \rangle \right|^2
%             \leq 2\sum^N_{\substack{j,k \\ \lambda_j + \lambda_k \neq 0}} \left| \langle j|H|k \rangle \right|^2
%             \leq N^2
% \end{equation}
