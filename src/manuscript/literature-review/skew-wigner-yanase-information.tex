\section{Косая информация Вигнера-Янасе}
\label{sec:skew-wigner-yanase-information}
% история
% определение
% применение
% свойства
% формула
% связь c информацией Фишера
% связь со вторым моментом

% 10.1103/PhysRevLett.91.180403 Inro
% В исследовании измерения квантовомеханических операторов Вигнер показал, что наблюдаемые, которые не коммутируют с аддитивной сохраняющейся величиной, труднее измерить, чем те, которые коммутируют с сохраняющейся величиной; то есть, наличие закона сохранения накладывает ограничение на измерение наблюдаемых, которые несовместимы (не коммутируют) с сохраняющейся величиной [1,2]. Араки и Янасе строго установили, что наблюдаемые величины, не совпадающие с сохраняющейся величиной, не могут быть измерены точно (в смысле фон Неймана), возможно только приближенное измерение, и существует компромисс между "размером" измерительного прибора и точностью измерения [3,4]. Это составляет знаменитую теорему Вигнера-Араки-Янасе, которая накладывает принципиальное ограничение на измерение квантовомеханических наблюдаемых. Эта теорема в дальнейшем исследовалась многими авторами [5-9].
% Cледовательно наблюдаемые величины,
% которые коммутируют с аддитивными сохраняющимися величинами
% (энергия, компоненты линейного и углового моментов, электрический заряд),
% могут быть измерены с помощью микроскопических аппаратов,
% а те, которые не коммутируют с этими величинами, требуют для своего измерения макроскопических систем\cite{Wigner1952, Araki1960}.

%
%Согласно квантовомеханической теории,
% екоторые наблюдаемые величины могут быть измерены гораздо легче, чем другие:

Араки и Янасе строго установили, что наблюдаемые величины,
которые не являются интегралами движения,
не могут быть измерены точно (в смысле фон Неймана),
и возможно только приближенное измерение.
Согласно знаменитой теореме Вигнера-Араки-Янасе, которая накладывает принципиальное ограничение на измерение квантовомеханических наблюдаемых,
существует компромисс между ``размером'' измерительного прибора и точностью измерения\cite{Araki1960, Yanase1961, Ozawa1991, Ozawa2002a, Ozawa2002b, Matsumoto1993, Kakazu3469}.
Наблюдаемые величины,
которые коммутируют с аддитивными сохраняющимися величинами
(энергия, компоненты линейного и углового моментов, электрический заряд),
могут быть измерены с помощью микроскопических аппаратов;
те же, которые не коммутируют с этими величинами, требуют для своего измерения макроскопических систем\cite{Wigner1952, Araki1960}.
Отсюда возникает проблема определения меры наших знаний относительно последних.

% Theoretical and Mathematical Physics, 202(1): 104–111 (2020)
Косая информация\cite{Wigner1963} была введена Вигнером и Янасе
в контексте квантовых измерений в качестве меры информации,
содержащейся в векторе квантового состояния, лежащего под углом по отношению к наблюдаемой $A$
%
\begin{equation}\label{eq:wyi}
  I_{WY}(\rho, A)
  = -\frac{1}{2} Tr([\sqrt{\rho}, A])^2
  = Tr(\rho A^2) - Tr(\sqrt \rho A \sqrt \rho  A ).
\end{equation}
%
В частности, если $\rho = \ket{\psi}\bra{\psi}$ --- чистое состояние, тогда
%
\begin{equation}\label{eq:wyi-pure}
  I_{WY}(| \psi \rangle, A)
  = \langle \psi | A^2 | \psi \rangle - \langle \psi | A| \psi \rangle ^2.
\end{equation}
Косая информация удовлетворяет условию выпуклости.
При объединении двух различных ансамблей информация о перекосе уменьшается, то есть
%
\begin{equation}\label{eq:wyi-convex}
  I_{WY}\p{\alpha \rho_1 + \beta \rho_2, A}
  \leq \alpha I_{WY}(\rho_1) + \beta I_{WY}(\rho_2),
\end{equation}
где $\alpha + \beta = 1$ и $\alpha, \beta \geq 0$.
%
Косая информация удовлетворяет условию аддитивности
%
\begin{equation}\label{eq:wyi-additivity}
 I_{WY}(\rho_1 \otimes \rho_2, A_1 \otimes 1 + 1 \otimes A_2)
 = I_{WY}(\rho_1, A_1) + I_{WY}(\rho_2, A_2),
\end{equation}
%
где $\rho_1$, $\rho_2$ матрицы плотности подсистем,
а $A_1$, $A_2$ соответствующие локальные наблюдаемые.
%

Косая информация Вигнера-Янасе существенно отличается от энтропии фон Неймана \cite{Wigner1952, Araki1960, Lieb1973prl, Lieb1973, Wehrl1978},
но глубоко связана с ней.
Энтропия, как ее обычно определяют, является мерой нашего незнания\cite{Weaver1949}.
Это мера, в которой знания о любой наблюдаемой величине находятся в одном ряду.
С точки зрения энтропии информационное содержание всех чистых состояний,
которые могут быть описаны одним вектором состояния, одинаково.
Это неверно для косой информации Вигнера-Янасе.

Позднее было признано, что косая информация,
помимо своего изначального значения как меры информационного содержания состояний,
допускает также несколько интерпретаций,
носящих более физический и теоретико-информационный характер.
Например, Шунь Лун Ло показал\cite{Luo2003prl, Luo2005, Luo2005pra, Luo2006, Luo2017},
что косую информацию Вигнера-Янасе можно интерпретировать
как квантовую неопределенность наблюдаемой $A$ для квантового состояния $\rho$.

Интригующей и тонкой особенностью косой информации является использование в ней квадратного корня из оператора плотности квантового состояния.
Тем не менее популярность получили и естественные модификации косой информации Вигнера-Янасе,
в которых нет корня из оператора плотности.
Например, величина
%
\begin{equation}\label{eq:wyi-modification-l}
  L(\rho, A)
  = -\frac{1}{2} Tr([\rho, A])^2.
\end{equation}
%
Значительным преимуществом величины $L(\rho, A)$ в сравнении с косой информацией является возможность ее измерения экспериментально~\cite{Girolami2014}.
В~\cite{Karpat2014} было показано,
что $L(\rho, A)$ можно измерить в "интерферометрической" установке (an interferometric setup) путем выполнения только двух программируемых измерений независимо от размерности квантовой системы.
Однако существует проблема зависимости $L(\rho, A)$ от характеристик вспомогательной системы, как было отмечено в~\cite{Yadin2016}.
С помощью сравнения косой информации с некоторыми ее естественными модификациями были раскрыты\cite{Luo2020} математические,
а также физические причины использования квадратного корня.

% Skew information has many nice properties and interpretations that make it useful in quantum information theory[1], [8].
В настоящее время косая информация Вигнера-Янасе нашла много применений в квантовой теории информации~\cite{Wigner1963, Luo2017}.
% For example, skew information can be used to construct measures of quantum correlations [9]–[11],
% to quantify quantum coherence [12]–[15], to quantify asymmetry [15], and so on.
Например, информация о перекосе может быть использована для построения мер квантовых корреляций~\cite{Luo2005, Luo2012, Li2016a, Sun2017},
для количественной оценки квантовой когерентности~\cite{Girolami2014, Yu2017, Luo2017, Luo2018},
для количественной оценки асимметрии~\cite{Luo2018} и так далее.
% It has also been used to study quantum phase transitions [16]–[20], uncertainty relations [6], [21]–[23], et cetera.
Она также использовалась для изучения квантовых фазовых переходов~\cite{Karpat2014, Malvezzi2016, Li2016b, Lei2016, Qiu2017}, соотношения неопределенностей~\cite{Luo2005},~\cite{Yanagi2005, Furuichi2010, Chen2016} и т.д.

Так как косая информация удовлетворяет условию выпуклости~(\ref{eq:wyi-convex}),
она принимает максимальное значение в случае,
когда $\rho = \ket{\psi}\bra{\psi}$ --- чистое состояние.
Оценка сверху величины косой информации может быть получена из выражения~(\ref{eq:wyi-pure})
%
\begin{equation}\label{eq:wyi-upper-bound}
  I_{WY}(\ket{\psi}, A)
  \leq  \bra{\psi} A^2 \ket{\psi}
  = \sum\limits_{i, j} \bra{\psi} A_i A_j \ket{\psi}
  \leq N^2,
\end{equation}
%
где $A_i$ --- локальная наблюдаемая $i$-ой подсистемы при условии,
что $A_i^2 = 1$.
Из свойств косой информации~(\ref{eq:wyi-additivity}) и~(\ref{eq:wyi-upper-bound}) следует,
что информация Вигнера-Янасе является частным
случаем обобщённой меры информации $F$~(см.~определение~\ref{def:f}),
и может быть использована для оценки количества запутанных частиц в системе.
% Chen2005


% ---- PRA
Тем не менее полноценное исследование многочастичной запутанности в системе
требует разработки соответствующих экспериментальных методов.
Основной недостаток квантовой информации Вигнера-Янасе заключается в том,
что эта величина не могла быть измерена экспериментально,
в отличии от информации Фишера\cite{Garttner2018}.
Одним из главных достижений данной работы является преодоление этого препятствия.
В Главе~\ref{chapter:wyi-mesuarement} представлена теория экспериментального метода определения величины косой информации Вигнера-Янасе
в спиновой системе с диполь-дипольным взаимодействием в многоквантовом эксперименте ЯМР.
Оказывается,
что в этом случае косая информация Вигнера-Янасе связана со значением
второго момента спектра интенсивностей многоквантовых когерентностей в системе.

% It is not a random magic because skew information may be identifier as a paradigmatic version of quantum Fisher information.
% % the statistical idea underlying the skew information is the Fisher information in the theory of statistical estimation.
