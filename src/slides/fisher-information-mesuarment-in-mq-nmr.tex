\begin{frame}{Определение информации Фишера в МК ЯМР (PRA-19)\footnote{S.I. Doronin, E.B. Fel'dman,  I.D. Lazarev, \textit{Phys. Rev. A}, \textbf{100}, 022330 (2019)}}
  $$ G(\tau, \phi) =
     \mathrm{Tr}\left\{
         e^{iH_\mathrm{MQ}\tau} e^{i\phi I_z} e^{-iH_\mathrm{MQ}\tau}
         \rho_\mathrm{eq}
         e^{iH_\mathrm{MQ}\tau} e^{-i\phi I_z} e^{-iH_\mathrm{MQ}\tau}
         I_z \right\}
  $$
  \begin{alertblock}{}
      Дисперсия распределения интенсивности МК когерентностей ЯМР определяет нижнюю границу информации
      Фишера\footnote[frame]{
      M. G\"arttner, P. Hauke, and A.M. Rey. \textit{Phys. Rev. Lett.} \textbf{120}, 040402 (2018)}:
      $$
      F_Q \geq 2M_2
      $$
  \end{alertblock}
  Однако формула справедлива только в том случае,
  если сигнал $G(\tau, \varphi)$ является \textit{out-of-time-ordered correlator} (OTOC).
  Для низких температур это условие не выполняется.
  Возможно обойти это ограничение,
  если усреднить сигнал МК эксперимента ЯМР по начальному состоянию:
  $$ G_\mathrm{LT}(\tau, \phi)
     = \mathrm{Tr}\left\{
       e^{iH_\mathrm{MQ}\tau} e^{i\phi I_z} e^{-iH_\mathrm{MQ}\tau}
       \rho_\mathrm{eq}
       e^{iH_\mathrm{MQ}\tau} e^{-i\phi I_z} e^{-iH_\mathrm{MQ}\tau}
       {\color{red} \rho_\mathrm{eq}}
    \right\}
  $$
\end{frame}


