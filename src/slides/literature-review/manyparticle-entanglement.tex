\begin{frame}{Многочастичная запутанность}
  \vspace{-5mm}
  $$
  \left| \Psi_{3-\mathrm{ent}} \right\rangle
  = \frac{1}{\sqrt{8}}
  \left| \uparrow_1 \right\rangle
  \otimes
  \left(
    \left| \uparrow_2\uparrow_4\right\rangle +
    \left|\downarrow_2\downarrow_4\right\rangle
  \right)
  \otimes
  \left(
    \left|\uparrow_3\uparrow_5\uparrow_9\right\rangle +
    \left|\downarrow_3\downarrow_5\downarrow_9\right\rangle
  \right)
  \otimes
  \left(
    \left|\uparrow_6\uparrow_7\uparrow_8\right\rangle +
    \left|\downarrow_6\downarrow_7\downarrow_8\right\rangle
  \right)
  $$
  % \begin{alertblock}{Определение многочастичной запутанности}
  \vspace{-5mm}
  \begin{alertblock}{}
    Чистое состояние $N$ частиц является $k$-частино запутанным, если
%
\begin{equation}
  \left| \Psi_{k-\mathrm{ent}} \right\rangle
  = \otimes^\mathrm{M}_{i=1} \left| \Psi_{i} \right\rangle,
\end{equation}
%
где $\left| \Psi_{i} \right\rangle$ - это многокубитное несепарабельное или однокубитное состояние подсистемы с $N_i$ частицами
$\left( \sum_{i=1}^N N_i = N \right)$,
и существует такое  $ m \in \mathbb{N}$, что $N_{m} \ge k$.
Смешанное состояние $\rho_{k-\mathrm{ent}}$ может быть представлено как
%
\begin{equation}
  \rho_{k-\mathrm{ent}} =
  \sum\limits_{l} p_l \ket{\Psi_{k_l-\mathrm{ent}}}\bra{\Psi_{k_l-\mathrm{ent}}},
  \end{equation}
%
и существует такое $l$, что $k_l \geq k$.

  %\begin{alertblock}{Многочастичная запутанность $k$-частиц}
  % $$
  % \left| \Psi_{k-\mathrm{ent}} \right\rangle
  % 	= \otimes^\mathrm{M}_{i=1} \left| \Psi_{i} \right\rangle,
  % $$
  % где $\left| \Psi_{i} \right\rangle$ --- состояние подсистемы с $N_i$ частицами,
  % $ \sum\limits_{i=1}^N N_i = N $
  % и $ \exists m: N_{m} \ge k$
  \end{alertblock}

\end{frame}
\note{
    До недавнего времени исследования были ограничены изучением запутанности
    и квантового дискорда между двумя подсистемами
    и направлены на определение мер этих величин.
    Вместе с тем более существенны не меры квантовых корреляций,
    а сам факт их наличия. 
    В данной работе используется классификация, 
    основанная на определении размера наибольшего запутанного кластера в некотором квантовом состоянии. 
    
    На слайде приведен пример трехчастично запутанного состояния.
    Очевидно, что максимальный размер запутанного кластера равен трем. 

    В общем случае наличие запутанности порядка $k$,
    говорит о том что существует несепарабельная подсистема размера $k$.
}

\begin{frame}{Меры многочастичной запутанности}
  \vspace{-2mm}
  \begin{block}{Свойства}
    \begin{enumerate}
      \item $
        F(\rho_1 \otimes \rho_2 ,H_1 \otimes I_2 + I_1 \otimes H_2)
    	= F_{1} (\rho_1, H_1) + F_{2} (\rho_2 , H_2)
      $
      \item $F \leq N^2$
    \end{enumerate}
  \end{block}
  \vspace{-1mm}
  \begin{alertblock}{Уточнение верхней границы величины $F$}
    % Пусть в системе максимальный размер несепарабельной подсистемы равен $k$, тогда
    $$ F^{k} \leq \left[ \frac N k \right] k^2 + \left(N - k \left[ \frac N k \right]\right)^2 $$
    Если нарушается неравенство, то размер запутанного кластера больше или равен k+1. 
  \end{alertblock}
  \vspace{-1mm}
  \begin{examples}%{Примеры}
    \begin{enumerate}
      \item Косая информация Вигнера-Янасе\footnote[frame]{Zeqian Chen \textit{Phys. Rev. A} \textbf{71}, 052302 (2005)}
      \item Квантовая информация Фишера\footnote[frame]{P. Hyllus et al. \textit{Phys. Rev. A} \textbf{85}, 022321 (2012)}
    \end{enumerate}
  \end{examples}
\end{frame}
\note{
 В настоящее время признано, что большинство физических процессов в природе можно сформулировать в терминах обработки информации, и концепция информации может быть центральной для понимания квантовой теории.

 Оказывается, что если некоторая мера F удовлетворяет двум свойствам:
 Первое это  аддитивность, что естественно для меры информации, 
 Второе это наличие верхней границы, 
 то есть  значение меры не превышает $N^2$, 
 где N число частиц в системе. 
 Тогда для состояний, 
 в которых максимальным размером несепарабельной системы является $k$,
 верхнюю границу можно уточнить. 

 Таким образом мы получаем неравенство типа Белла. 
 Если при вычислении меры F некоторое состояние нарушает неравенство при фиксированом $k$,
 это означает что состояние является $k+1$ частично запутанным. 

 В качестве меры F могут быть, например, косая информация Вигнера-Янасе и квантовая инфомрация Фишера. 
 Для доклада нам не потребуются оригинальные определения этих информаций,
 но я кратко этот вопрос освещу. 
}


% \begin{frame}{Меры многочастичной запутанности}
% Фиксируем какое-то значение $k$ - максимальное количество спинов в несепарабельной подсистеме
% $$ N > k \geq N_1 \geq N_2 \geq \dots \geq N_l, \quad \sum_{i=0}^{l} N_i = N, $$
% где $l$ количество несепарабельных подсистем.
%
% Верхняя граница информации Фишера для такой системы:
% $$
% F^k = F_{1} + \dots + F_{l}
% \leq N^2_1 + \dots + N^2_e
% \leq \underbrace{
% 	k^2 + k^2 + \dots + k^2
% 	}_{m = \left[\frac N k \right] \, \mbox{раз}}
% 	+ \underbrace{(N-km)^2}_{\mbox{остаток}}
% $$
%
% \begin{alertblock}{}
%     $$  F^k \leq \left[ \frac N k \right] k^2 + \left(N - k \left[ \frac N k \right]\right)^2 $$
% \end{alertblock}
% \end{frame}
% \note{
% Доказательства этого факта сводится к вычислению верхней границы квантовой информации Фишера для с фиксированым размером не сепарабельных подсистем.
% Мы для себя фиксируем какое-то значение $k$ - максимальное количество спинов в несепарабельной подсистеме. Сделаем оценку сверху для информации Фишера.
%
% Пусть есть я ...
%
% Рассмотрим пару подсистем для которых верно $k > N_i > N_{i-1} > 0$.
% Квантовая информации Фишера этих двух подсистем:
% $$
% F_{Q\,i,i-1} \leq N_{i}^2 + N_{i-1}^2
% $$
% При этом если мы рассмотрим случай когда в правой системе на один спин меньше, а в левой на одим больше то оценка верхней границы будет выше:
% $$
% \bar F_{Q\,i,i-1}
% 	\leq \left( N_{i}^2 + 1 \right)
% 		+ \left( N_{i-1}^2 - 1\right)
% 	\leq N_{i}^2 + N_{i-1}^2
% 		+ 2 \left(N_{i} - N_{i-1} + 1 \right)
% $$
% Следовательно в худшем случае...???
%
% Следовательно если мы подсчитаем информацию Фишера, то мы
% можем дать нижнюю границу количества запутанных частиц.
% }
