\begin{frame}{Диагонализация гамильтониана}
Многоквантовый гамильтониан
$$
H_\mathrm{MQ} = - \dfrac{D}{4} \left\{
    \left( I^{+} \right)^2 + \left( I^{-} \right)^2
\right\}
$$
коммутирует с проектором квадрата полного углового момента
$ \left[ H_\mathrm{MQ}, \hat I^2 \right] = 0 $.
Следовательно, в базисе собственных значений  $\hat I^2$ и $I_z$  гамильтониан имеет блочно-диагональный вид :
$$
H_\mathrm{MQ} = \mathrm{diag} \left\{
    H_\mathrm{MQ}^{N/2},
    H_\mathrm{MQ}^{N/2 - 1},
    \dots
    H_\mathrm{MQ}^{N/2 - [N/2]}
\right\},
$$
где каждый блок соответствует собственному значению полного углового момента.
\end{frame}
\note{
    Так как в нашей модели все констаны одинаковые, МК гамильтониан упрощается.
    Этот гамильтониан коммутирует ...слайд...
}