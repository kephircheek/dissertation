\begin{frame}{Косая информация Вигнера-Янасе}
  \begin{block}{}
    Косая информация\footnote[frame]{
      E. P. Wigner and M. M. Yanase,
      \textit{Proc. Nat. Acad. Sci. USA},
      \textbf{49}, 910–918 (1963)
    }
    была введена Вигнером и Янасе в контексте квантовых измерений в качестве меры информации,
    содержащейся в векторе квантового состояния, лежащего под углом по отношению к наблюдаемой $A$.
  \end{block}
  $$
    I_{WY}(\rho, A)
    = -\frac{1}{2} Tr([\sqrt{\rho}, A])^2
    = Tr(\rho A^2) - Tr(\sqrt \rho A \sqrt \rho  A )
  $$
  Для чистых состояний:
  $$
    I_{WY}(| \psi \rangle, A)
    = \langle \psi | A^2 | \psi \rangle - \langle \psi | A| \psi \rangle ^2
  $$
  \vspace{-5mm}
  \begin{block}{}
    Можно интерпретировать\footnote[frame]{S. Luo, \textit{Phys. Rev. Lett.} \textbf{91}, 180403 (2003)}
    косую информацию
    как квантовую неопределенность наблюдаемой $A$ для квантового состояния $\rho$.
  \end{block}
\end{frame}
\note{
    Косая информация была введена Вигнером и Янасе в контексте квантовых измерении.
    Позднее было признано, что косая информация,
    помимо своего изначального значения как меры информационного содержания состояний,
    допускает также несколько интерпретаций,
    носящих более физический и теоретико-информационный характер.

    В настоящее время эта величина нашла много применений в квантовой информации.

    Интригующей и тонкой особенностью косой информации является использование в ней квадратного корня из вектора квантового состояния (оператора плотности).

    Но тут

    % Skew information was introduced by Wigner and Yanase in the context of quantum measurements.
    % It is the quantity as a measure of the information content of the quantum state $\rho$ \textit{skew} to the observable $H$.
    % Here, \textit{tr} denotes the operator trace, and brackets denotes the commutator between operators.
%
    % Not that skew information is quite different from, but deeply related to, the ubiquitous von Neumann entropy [1]–[4].
    % Alse from a dual standpoint, we can interpret skew information
    % as the quantum uncertainty of the observable $A$ in the quantum state $\rho$ [5]–[8].
%
    % Skew information has many nice properties and interpretations that make it useful in quantum information theory[1], [8].
    % For example, skew information can be used to construct measures of quantum correlations [9]–[11],
    % to quantify quantum coherence [12]–[15], to quantify asymmetry [15], and so on.
    % It has also been used to study quantum phase transitions [16]–[20], uncertainty relations [6], [21]–[23], et cetera.
%
    % However such investigations like many-particle entanglement require the development of corresponding experimental methods.
    % In particular, it was shown [6] that a lower bound on the quantum Fisher information coincides with the double
    % second moment of the spectrum of multiple quantum (MQ) coherences.
    % % M. G\"arttner, P. Hauke, and A.M. Rey.Phys. Rev.Lett. 120, 040402, 2018
%
    % Bellow we demonstrate that the Wigner–Yanase information in a spin system (s = 1/2) with the dipole–dipole interactions
    % (DDI) in the MQ NMR experiment also connected to second moment of the spectrum of multiple quantum
    % (MQ) coherences.
%
    % It is not a random magic because skew information may be identifier as a paradigmatic version of quantum Fisher information.
    % % the statistical idea underlying the skew information is the Fisher information in the theory
    % % of statistical estimation.
}%