\begin{frame}{Ацаркин Вадим Александрович}
\begin{enumerate}
    \item В качестве критического замечания отмечу, что в диссертации, на мой взгляд, недостаточно подробно обсуждены конкретные преимущества использования именно многочастичной запутанности в квантовых вычислениях. Какие реальные перспективы открывает такой подход, имеются ли уже достоверные оценки – или же это дело отдаленного будущего? 

    \item Сходная проблема возникает и с низкими температурами. В ряде случаев теоретически изученные диссертантом температурные диапазоны (милликельвины) пока малодоступны. Понятно, что эти исследования устремлены в будущее, это похвально и необходимо. Тем не менее, хотелось бы более четко представлять себе перспективу. 
\end{enumerate}

\end{frame}

\begin{frame}{Погосов Вальтер Валентинович}
\begin{enumerate}
    \item При рассмотрении многоспиновой запутанности в квазиодномерных цепочках в главе 4 используется гауссово приближение для распределения интенсивностей многоквантовых когерентностей. Остается неясным, насколько оправдано это приближение и как выход за его рамки может поменять результат.

    \item В главе 5 на рис. 5.1 приведены зависимости оценки снизу числа запутанных спинов от обратной температуры на основе квантовой информации Фишера и на основе косой информации Вигнера-Янасе. Несмотря на качественное согласие между этими результатами имеются существенные количественные отличия. С чем они связаны, и которая из оценок более адекватна? 

    \item В диссертации практически отсутствует качественное рассмотрение. 
\end{enumerate}
\end{frame}

\begin{frame}{Цуканов Александр Викторович}
\begin{enumerate}
    \item Автор классифицирует запутанность многочастичных систем по количеству запутанных частиц. Возможна ли дальнейшая классификация запутанности при уже зафиксированном числе запутанных частиц? Например, для описания парной запутанности существует параметр согласованности, который варьируется от 0 до 1 и позволяет не только установить факт запутывания двух частиц, но и степень их запутанности (0 – частицы не запутаны, 1 – частицы запутаны максимально). 

    \item Рассмотрена фермионная система частиц с полуцелым спином. Адаптируется ли данная схема для бозонов – фотонов или атомных частиц с целым спином? 
\end{enumerate}

\end{frame}